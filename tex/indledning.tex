\section{Indledning}

I følge \citeauthor{hansbolKonstruktionAfProfessionel2008} er der en gruppe erhverv, der i stigende grad søger og kæmper for anerkendelse som professioner i det 21. århundrede — herunder pædagogerne (\citeyear{hansbolKonstruktionAfProfessionel2008}, s. 19).
For pædagogerne har denne professionskamp været lang, og præget af diffuse krav, både til arbejdets udførelse og uddannelsens karakter \autocite[ss. 48-51]{kofodBornepolitikkenOgUdviklingen2007}.
I lighed med andre “relationsprofessioner” \autocite{moosRelationsprofessionerLaererePaedagoger2008}, “velfærdsprofessioner” \autocite{frederiksenVelfaerdsprofessionerMellemOmsorg2017} eller sågar “semiprofessioner”\autocite{kofodBornepolitikkenOgUdviklingen2007} \todo{undersøg kilde}, kommer pædagogerne til kort i forhold til traditionelle minimumskrav for professioner:
Uddannelse til professionen foregår ikke i en videnskabelig institution; og der er heller ikke et anerkendt monopol på, at kun pædagoger kan levere pædagogisk arbejde og pædagogiske ydelser \autocite[s.53]{kofodBornepolitikkenOgUdviklingen2007}.
På det specialiserede voksenområde udgør pædagogerne kun omkring en tredjedel af arbejdsstyrken \autocite[ss. 8-9]{kommunerneslandsforeningFaktaOmKommunernes2019}.
Hertil kommer de andre faggrupper — ergoterapeuter, sosu-hjælpere og -assistenter, der er ansat på dette område, samt en meget stor gruppe af ufaglært arbejdskraft \footnote{Den præcise andel fremgår ikke af rapporten - der henvises til en gruppe af “øvrige” på omkring halvdelen \autocite[s. 8]{kommunerneslandsforeningFaktaOmKommunernes2019}.}.

Spørgsmålet omkring, “hvori den professionelle identitet består” blandt de erhvervsgrupper, der primært arbejder i den offentlige sektor, skriver \citeauthor{hansbolKonstruktionAfProfessionel2008} ind i en definitionskamp omkring den offentlige sektors formål i overgagnen fra velfærdsstaten til velfærdssamfundet.
latex\autocite[s. 19]{hansbolKonstruktionAfProfessionel2008}.

Begrebet professionsidentitet har en særlig betydning i det moderne samfund, hvor der i stigende grad forventes, at kunne anvende sin egen personlighed i opfyldelsen af de krav, der stilles af professionen \autocite{hansbolKonstruktionAfProfessionel2008}. I følge \citeauthor{mik-meyerIndledningSkabeProfessionel2012} ender den professionelle med, at se sig selv og sit arbejde subjektivt, hvori man finder professionsidentiteten \autocite[s. 458]{mik-meyerIndledningSkabeProfessionel2012}.

Den enkelte pædagog, som repræsentant for sin profession, forventes at forholde sig til faglige traditioner; men også forholde sig til, at disse er til stadighed mere omskiftelige og mangfoldige \autocite{hansbolKonstruktionAfProfessionel2008}\todo{sidetal}. I følge \citeauthor{hansbolKonstruktionAfProfessionel2008}, er det netop her en “personligt forankret professionstil” er en fordel (\citeyear{hansbolKonstruktionAfProfessionel2008}, s. 33).

\citeauthor{kofodOrganisationOgLedelse2016} tager afsæt i \citeauthor{baumanLiquidModernity2000} i en beskrivelse af pædagogiske organisationer. 
Et af \citeauthor{baumanLiquidModernity2000}s pointer er, at arbejdsmarkedet i den flydende modernitet er præget af stor usikkerhed, der stiller store krav til medarbejdernes fleksibilitet \autocite[s. ??]{baumanLiquidModernity2000}\todo{sidetal}.
Heller ikke (special)pædagogiske arbejdspladser er immune overfor den flydende modernitets usikkerheder.
Både økonomiske konjuktursvingninger og en reduceret loyalitet til den enkelte arbejdsplads, medfører fx, at der er svagere tilknytning mellem medarbejder og institution, sammen med krav om evig omsitillingsevne \autocite[s. 166f]{kofodOrganisationOgLedelse2016}.

Men hvordan tager alt dette sig ud i praksis?

\section{Problemformulering}
Jeg vil gerne undersøge, hvordan pædagogiske praktikere forholder sig til deres professionsidentitet.
Herunder er der en række underspørgsmål:

Hvordan oplever pædagoger i dag deres faglige særegenhed?

Hvordan forholder de sig til de mange andre erhvervsgrupper og fagligheder de er omgivet af?

Hvad med de andre interessenter, der forsøger at definere og give formål til pædagogers praksisforståelse?

Udtrykkes der en professionsidentitet, der er forankret i og udgør en del af de udøvende pædagogers personlighed og selvopfattelse, som der står ovenfor?
I så tilfælde, hvordan?

Oplever de, noget der giver grundlag for fællesskab på arbejdspladsen?

\subsection{(Special)pædagoger som profession}
I en hverdagsopfattelse af professioner er velfærdsprofessionerne — herunder socialpædagoger — tilsyneladende udmærkede professioner.
Man har fx tilegnet sig (særlige) kompetencer, og lever af at benytte disse, på en standariseret facon. \autocite[ss. 443-445]{frederiksenVelfaerdsprofessionerMellemOmsorg2017}.

Sådan en hverdagsopfattelse af professioner er dog mangelfuld, som \citeauthor{frederiksenVelfaerdsprofessionerMellemOmsorg2017} beskriver. Den siger fx ikke noget om hvor standarden for arbejdets udførelse opstår; eller hvad “optagelseskriterierne” er \autocite[s. 445]{frederiksenVelfaerdsprofessionerMellemOmsorg2017}.
For en mere præcis beskrivelse af professioner, kan man henvende sig til professionssociologien. I følge \citeauthor{frederiksenVelfaerdsprofessionerMellemOmsorg2017}, har professionsforskningen samlet sig omkring tre retninger \autocite[s. 445]{frederiksenVelfaerdsprofessionerMellemOmsorg2017}:
\begin{itemize}
  \item
    taksonomiske professionsbegreber
  \item
    funktionalistisk professionsforståelse
  \item
    kynisk professionsforståelse
\end{itemize}

Jeg vil i det følgende gennemgå disse retninger, som beskrevet af \citeauthor{frederiksenVelfaerdsprofessionerMellemOmsorg2017}, og hvordan pædagogisk arbejde, i lighed med de andre velfærdsprofessioner, ikke helt når op til at være fuldbyrdede professioner.

\subsubsection{Træk ved professioner}
I den taksonomiske professioneforståelse forsøger man at beskrive kendetegn ved professioner, for derefter at se om en erhvervsgrupper lever op til disse træk.
Gør den det, er den for profession at regne \autocite[s.446]{frederiksenVelfaerdsprofessionerMellemOmsorg2017}.

\citeauthor{frederiksenVelfaerdsprofessionerMellemOmsorg2017} henviser her til \citeauthor{molanderProfesjonsstudierIntroduksjon2008}, der deler en professions kendetegn op i \textit{organisatoriske} og \textit{performative} kategorier (\citeyear{frederiksenVelfaerdsprofessionerMellemOmsorg2017}, s. 446).

De organisatoriske aspekter ved professioner tilsiger, at en erhvervsgrupper der har kontrol over sine arbejdsopgaver, er en profession. Disse indbefatter \autocite[s. 18ff]{molanderProfesjonsstudierIntroduksjon2008}:
\begin{itemize}
  \item
    Monopol
  \item
    Autonomi
  \item
    Politisk konstituering
  \item
    Institutionelt imperativ
  \item
    Professionel sammenslutning
\end{itemize}

De performative aspekter ved professioner omhandler hvordan en formaliseret kompetence eller kunnen kombineres med skøn, og dermed udgør praksis. De består af \autocite[s 19ff]{molanderProfesjonsstudierIntroduksjon2008}:
\begin{itemize}
  \item
    Der ydes tjenester
  \item
    Til klienter
  \item
    hvor der løses prakiske problemer
  \item
    Tjenesterne er ændringsorienterede
  \item
    og der anvendes viden
  \item
    sammen med skøn
  \item
    Tjenesterne er normativt regulerede
  \item
    Praksis er præget af usikkerhed, der er den professionelles ansvar
\end{itemize}

Pædagoger i det specialiserede voksenområde opfylder mange, men ikke alle af disse kriterier.
De har, som nævnt ovenfor, ikke monopol på at udøve (special)pædagogisk arbejde.
Der er også kun en delvis autonomi, idet der er flere instanser, der dikterer arbejdets udførelse og formål. \todo{bedre forlulering. og en kilde.}
Arbejdet er politisk konstitueret i kraft af \autocite{sociallatex-ogindenrigsministerietBekendtgorelseAfLov2019}, og der er et institutionelt imperativ i, at sikre at også de svageste i samfundet har et godt og værdigt liv.
Socialpædagogerne er organiseret i Socialpædagogernes Landsforening.

Tjenesten, der leveres, er støtte til daglig livsførsel for de handicappede.
Det praktiske problem, der løses, indebærer udvikling af færdigheder og minimering af ubehag \todo{noget af en påstand. skal uddybes og kilde på}; der indebærer en tilstandændring.
Der anvendes (i et vist omfang) specialiseret viden om specifikke problemstillinger, med skøn i forhold til den individuelle borgers aktuelle problemstillinger.
Der er normer for, hvad der anses for en værdig tilværelse og det gode liv; men pædagogen skal tage ansvar for, at kunne handle fejlbarligt.

Skoen trykker især omkring monopol, autonomi og vidensanvendelse, hvilket \citeauthor{frederiksenVelfaerdsprofessionerMellemOmsorg2017} påpeger, er kendetegn ved alle velfærdsprofessioner.

\subsubsection{Professionernes funktion}\todo{Overflødigt afsnit?}
Ud fra dette bud på en taksonomi over professioner kommer pædagogprofessionen til kort på nogle omráder.
Den taksonomiske tilgang til professionsteori kan heller ikke beskrive, \textit{hvorfor} professioner er til, og indehar bestemte kendetegn \autocite[s. 450]{frederiksenVelfaerdsprofessionerMellemOmsorg2017}.

\citeauthor{frederiksenVelfaerdsprofessionerMellemOmsorg2017} henviser til Parsons, og hans funktionalistiske professionsteori, for et svar på “hvorfor professioner?”.
Professionerne har en samfundstjenlig funktion: de sikrer samfundets harmoni og moralske sammenhængskraft.
Dette kræver de ovennænte træk ved professionerne --- uden disse, kan professionerne ikke opfylde deres samfundsmæssige funktion \autocite[s. 450]{frederiksenVelfaerdsprofessionerMellemOmsorg2017}

Det er også i denne funktionopfyldelse man skal se professionernes status, i følge \citeauthor{frederiksenVelfaerdsprofessionerMellemOmsorg2017}.
Det ansvar, der ligger i at handle på samfundtes vegne og ikke efter eget forgodtbefindende, skal også aflønnes\todo{belønnes?}.
Og for at sikre, at dette sker, er en særlig uddannelse og et særligt normsæt påkrævede \autocite[s. 451]{frederiksenVelfaerdsprofessionerMellemOmsorg2017}.

Dette fordrer dog, at der er en grundlæggende konsensus i, hvad “samfundets bedste” er, og denne forklaringsmodel kommer også til kort, når der kan observeres konflikt og uenighed \autocite{frederiksenVelfaerdsprofessionerMellemOmsorg2017}.

\subsubsection{Afgrænsning og social lukning}

Profession og faglighed er også et tilbagevendende emne i Socialpædagogen \autocite[fx]{petersenHvadSigerEksperten2019}, hvor der eftersøges en særlig social- eller specialpædagogisk “faglighed” og “professionalisme”.
Hvad er det en pædagog på et specialpædagogisk opholdssted kan, som en ergoterapeut, en sosu-assisstent eller en ufaglært vikar ikke kan?

Heri er en anden form for kamp.
De forskellige fagligheder i personalegruppen stiller forskellige perspektiver på hvad, der udgør “godt arbejde”, og hvordan dette kan legitimeres som gyldig og professionel praksis.
Der synes også at være bred enighed blandt pædagogerne på Socialpædagogens fasebookegruppe om, at de kan noget særligt og værdifuldt \autocite{petersenSlagsMenneskeligAltmuligmand2019}, der er adskilt fra de omkringgivende fagligheder.
Denne kamp udspiller sig både lokalt, på den enkelte arbejdsplads, og i det specialpædagogiske felt i almindelighed.

Dette er et grundlæggende konfliktfylt syn på professioner, der står i kontrast til den konsensusforståelse vi ser ovenfor.

Dette forklarer \citeauthor{frederiksenVelfaerdsprofessionerMellemOmsorg2017} ud fra et kynisk perspektiv på professioner.
Status og anerkendelse er noget,der erobres.
Ved at danne en eksklusiv gruppe, der har monopol på udførelsen af et arbejde, bliver professionen bedre stillet til at forsvare sin position.
Den mest anvendte strategi for at opnå social lukning er, at kræve en bestemt uddannelse for at kunne blive optaget i professionen \autocite[s. 451ff]{frederiksenVelfaerdsprofessionerMellemOmsorg2017}.

Dette er, som vist overfor, ikke entydigt lykkedes for pædagogerne.
Det er især den store andel ufaglærte, der har tilgang til det samme arbejdsmarked, der står som elefanten i rummet.\todo{findes der et dansk idiom?}
Der er også en stigende andel pædagogiske assistenter, der forsøger at tilkæmpe sig lignende rettigheder — uden at have opfyldt de samme krav til optagelse i professionen.\todo{kilde. og relevans?}

\subsection{Pædagogerne — og de andre}

Som beskrevet i indledningen, er pædagogerne ikke de eneste erhvervsgrupper, der er beskæftiget på det specialpædagogiske område.
Jeg har beskrevet, hvordan dette udfordrer monopolet på at udføre pædagogisk arbejde.
Men det gør det også nødvendigt for pædagogerne, at forholde sig til, hvordan de adskiller sig fra eksempelvis sundhedsuddannede kolleger, der udgør omkring en tiendedel af de ansatte på det specialpædagogiske område \autocite[s. 8f]{kommunerneslandsforeningFaktaOmKommunernes2019}.

\citeauthor{porsKerneloseKerneopgaverSkolen2015} beskriver, hvordan moderne velfærdsorganisationer kan ses som “potentialitetsafsøgende organisationer”, der hele tiden er afsøgende efter, hvad de forskellige professioner kan tilbyde \autocite[s 310]{porsKerneloseKerneopgaverSkolen2015}.
Med afsæt i begrebet “kerneopgave” som noget fælles tredje, udenfor de enkelte faglige perspektivers gebet, bliver det pålæggende de forskellige fagliheder at byde sig til — og samtidig være åbne for, at de andre faggrupper har noget at byde ind med.
På denne måde bliver kampen for status og position gjort tandløs, og professionerne tvinges til at tage stilling til sin egen arbejdsudsførelse \autocite[s. 311 ff.]{porsKerneloseKerneopgaverSkolen2015}.

Denne evige afsøgning gør også ansvarsområdet mere diffust. En specialpædagog skal ikke blot forholde sig til den enkelte borgers umiddelbare vanskeligheder, men til dennes livskvalitet i helhed.
Dette understreges af \citeauthor{socialpaedagogerneSocialpaedagogiskeKernefaglighed2015} skrivelse omkring \citetitle{socialpaedagogerneSocialpaedagogiskeKernefaglighed2015}; hvor der undrestreges, at socialpædagoger er ”den helhedsorienterede specialist” ved målrettet relationsarbejde \autocite{socialpaedagogerneSocialpaedagogiskeKernefaglighed2015}.

Dette diffuse ansvarsområde stiller også krav til den professionelle pædagogs følelsesmæssige engagement. Relationsarbejdet fordrer, at pædagogen tager brug af sin personlighed for at skabe kontakt til borgeren — men på et professionelt niveau. Dette understeger den professionelle subjektivering, idet man i stigende grad forventes at bruge hele sig selv i sit arbejde \autocite[s. 71f]{mik-meyerIndledningSkabeProfessionel2012}.

\subsection{Omsorg på det åbne marked}

Velfærdsprofessionerne kendetegnes ved, at yde \textit{omsorg}, hvor de overtager opgaver, der før var forbeholdt familien \autocite{frederiksenVelfaerdsprofessionerMellemOmsorg2017}. 

Sammen med en stigende markedsøkonomisk orientering, hvor velfaærdssamfundets bruger søger hjælp og vejledning til sit eget selvrealiseringsprojekt; modsat velfærdsstatens klient der hendvendte sig til eksperten \autocite{hansbolKonstruktionAfProfessionel2008}.

Men hvem er egentlig “kunden” på det specialpædagogiske område?
De handicappede besidder ikke kompetencer, til selv at søge deres frihed; det ligger dermed i pædagogernes (selvbestaltede) kerneopgave, at frigøre dem \autocite{socialpaedagogerneSocialpaedagogiskeKernefaglighed2015}

omsorgsproblemet for velfærdsprofessionerne - der fordrer, at man tager arbejdet til sig; og gør den til en del af ens selvfortælling.

\subsection{Professionsidentitet i diffusion}
afrunde, og bring ind til kernen: professionsidentiteten.
Hvirdan pávirker det professionidentiteten, at 

professionens status er diffus, lige som opgaven er det

der er en evig forhandling og afsøgning overfor andre faggruooer - samtidig med en afgreænding og en differentiering

kontroldimensioen medfører, en normativ begrænsing af arbejdet
Omsorgperspektivet gør, at man öwrtager en personlig identifiaktion med arbejded (ikke det samme som identitet!)


\section{Metodiske overvejelser}

\citeauthor{molanderProfesjonsstudierIntroduksjon2008} taler om, at  professionsstudier kan foretages på makro- og mikroniveau.
Idet jeg vil vide noget om, hvordan individuelle praktikere forstår og forvalter deres professionsidentitet, er denne undersøgelse primært på mikroniveau (\citeyear{molanderProfesjonsstudierIntroduksjon2008}, s. 24).
Jeg vil nu kort ridse op, min tilgang til at besvare min problemformulering, efterfulgt af en gennemgang af mine metodiske valg.

For at få de udøvende pædagoger i tale har jeg arrangeret et gruppeinterview med to pædagoger, der arbejder på et botilbud for voksne med kognitive funktionsnedsættelser.
Begge pædagoger har lang arbejdserfaring.
Jeg håber på denne måde, at få et perspektiv på (special)pædagogisk faglighed og professionalisme, der er forankret i fagets traditioner.

Min analyse vil tage udgangspunkt i Foucault, og hans beskrivelser af mekanismer, der bidrager til social (selv)styring.
I tråd med den subjektivere(n)de professionsidentitet, kan man så se skjulte magtstrukturer i, hvordan man opretholder en personlig, professionel identitet som pædagog?\todo{Mitchell Dean?}

Dette vil jeg perspektivere til \citeauthor{baumanLiquidModernity2000}, og hans begreb om nutidens flydende modernitet (\citeyear{baumanLiquidModernity2000}).
Jeg tager afsæt i \citeauthor{kofodOrganisationOgLedelse2016}, der beskriver, hvordan \citeauthor{baumanLiquidModernity2000} kan hjælpe os til at forstå, de krav, der stilles til moderne pædagogiske arbejdspladser \autocite{kofodOrganisationOgLedelse2016}.
Dette fordi, \citeauthor{molanderProfesjonsstudierIntroduksjon2008} påpeger at en mikroanalyse også vil skulle referere til de kollektive forståelser og forventninger der kendetegner makroniveauet \autocite[s. 24]{molanderProfesjonsstudierIntroduksjon2008}.
\todo{dette kan gøres bedre}

\subsection{Den italesatte livsverden}
\citeauthor{tanggaardInterviewetSamtalenSom2015} beskriver interviewet som

\begin{quote}
en meget almindelig måde, at opnå viden om menneskers livssituation, deres meninger, holdninger og oplevelser} \autocite[s. 29]{tanggaardInterviewetSamtalenSom2015}.
\end{quote}

De beskriver videre, at interviewet kan anskues som en af midlerne i den “igangværende individuelle historiefortælling” omkring “selvet og det gode liv” \autocite[s. 30]{tanggaardInterviewetSamtalenSom2015}.
Jeg ønsker netop tilgang til mine informanters oplevelser af deres livsverden — deres umiddelbare oplevelser, og ikke mindst, hvordan de italesætter dem \autocite[s. 31]{tanggaardInterviewetSamtalenSom2015}.
Ud fra et socialkonstruktionistisk udgangspunkt vil jeg her se på, hvordan særlige elementer fremhæves som betydningsfulde.\todo{kilde? og uddybes}

\subsubsection{Hvis livsverden?}
Min empiri består af et gruppeinterview med to erfarne pædagoger.
De fremstår meget homogene — begge er kvinder, begge er over halvtreds, og begge har lang erfaring i specialpædagogisk arbejde.
Dette påvirker givetvis bredden i mit empiriske materiale.
Det kunne især have været ønskeligt, at haft nogle informanter, med en kortere tilknytning til pædagogfaget.
Desværre var der frafald ved interviewtidspunktet, der forhindrede dette.
Jeg har dermed ikke opnået, hvad \autocite{tanggaardInterviewetSamtalenSom2015} beskriver som idealet — en “mætning” af information omkring pædagogers forholden til deres professionsidentitet \autocite[s. 32]{tanggaardInterviewetSamtalenSom2015}.

\subsection{Individet i det postmoderne samfund}

\subsubsection{Individet som forbruger}
\citeauthor{baumanLiquidModernity2000} beskriver i sin udlægning af det postmoderne samfund en overgang fra modernitetens faste tryghed til nutidens flydende modernitet \autocite[s. 2]{baumanLiquidModernity2000}.
Samfundets institutioner mister sin autoritet, og det enkelte individ bliver moralsk ansvarligt for sig selv \autocite[s. 64ff]{baumanLiquidModernity2000}.
En del af dette ansvar i det flyd

Individerne i den flydende modernitet er, fremfor alt, \textit{forbrugere} \autocite[s 73ff; s. 76]{baumanLiquidModernity2000}.
Gennem forbruget køber de sig til identitetsmarkører --- herunder færdigheder til forsørgelse og midler til at overbevise arbejdsgivere at man fortjener en ansættelse \autocite[s. 74]{baumanLiquidModernity2000}.
I følge Bauman er det moderne forbrug ikke funderet i hverken behovsopfyldelse eller indfrielse af lyster - men i opfyldelse af ønsket.
Det er også heri Bauman ser betingelsen for individets frihed --- friheden til “at have identitet” er bundet op en afhængighed til shopping \autocite[s. 84]{baumanLiquidModernity2000}.

Også på arbejdsmarkedet kan det forbrugende individ shoppe rundt.
Hvis en af parterne ikke oplever, at ens ønsker og drømme opfyldes på sin arbejdsplads, kan man opløse forholdet og opsøge et nyt, der indehar mere “attraktive” og “spændende” kvaliteter.\todo{kilde. Kofod?}
Bauman anfægter, at arbejde i sig selv kan udgøre en forankring for selvforståelse og identitet.
Arbejde i det flydende moderne måles på, hvorvidt det er tilfredsstillende i sig selv; ikke hvor samfundstjenligt det måtte være \autocite[s. 139; 163f]{baumanLiquidModernity2000}.
Men denne frihed medfører også usikkerhed; der i følge Bauman også virker individuerende; i og med at det bryder med den solide modernitets forestillinger om sammenhold og solidaritet \autocite[s. 148]{baumanLiquidModernity2000}.

\citeauthor{baumanLiquidModernity2000} henviser til Robert Reich, der har en grov typologi over dagens arbejdsstyrke (\citeyear{baumanLiquidModernity2000}, s. 152):

Symbolanalytikerne; der opfinder og formidler;

Dem, der står for arbejdsstyrkens reproduktion;

Dem, der er beskæftiget med ansigt-til-ansigt personlig service;

Og “rutinearbejderne”; dem, der gør 'alt det andet'

Denne typologi tages også op af \citeauthor{kofodOrganisationOgLedelse2016}, der ser på Bauman i forhold til pædagogisk ledelse.
I hans udlægning lægges pædagogerne ind i gruppe 2 - “velfærdsstatens funktionærer”.
Men samtidig lægges den sociale sektors ansatte i den tredje gruppe \autocite[s. 166]{kofodOrganisationOgLedelse2016}.

For mig at se understreger dette specialpædagogikkens identitetsproblem.
Man ivaretager og opretholder samfundets ærdier på vegne af velferdsstaten; men er til tider svære at skille fra, de ufaglærte rutinearbejdere, der også udgør en 


\subsubsection{Individuerende selvstyre}
Bauman - panopticon → synopticon (s 3ish)/Hvad kan jeg gøre i stedet for hvordan gør jeg min pligt s 61

Foucault - governmentalty/subjektet skabes i og af den moderne stat/gør det du bør gøre af egen vilje

\subsection{Andet arbejde på området}
\citeauthor{nielsenAttraktivPaPapiret2017} har undersøgt hvordan mødet med den pædagogiske praksis er for nyuddannede pædagoger. De beskriver noget brat overgang og uoverensstemmmelse mellem en tiltagende akademiseret pædagoguddannelse og et relationelt forankret praksisfelt \autocite{nielsenAttraktivPaPapiret2017}.\todo{mere kød på?}

I \citetitle{dreyerespersenBekymrendeIdentiteterAnbragte2010} beskriver \citeauthor{dreyerespersenBekymrendeIdentiteterAnbragte2010} med henvisning til Goffmanns institutionsanalyse hvordan en oplevelse af særlige arbejdsmiljømæssige vilkår giver anledning til en speciel form for sammenhold og modstandsstrategier i en personalegruppe \autocite{dreyerespersenBekymrendeIdentiteterAnbragte2010}.

\citeauthor{hurFrigorelsensMagt2015} beskriver i \citetitle{hurFrigorelsensMagt2015}, hvordan udviklingen i handicapsynet og handicappolitiken har gjort grænserne for den specialpædagogiske faglighed mere diffus på nogle områder, samtidig med, at den introducerer nye styringsmekanismer overfor de handicappede \autocite{hurFrigorelsensMagt2015}.
Med afsæt i Goffmanns teatermetafor beskriver hun, hvordan udviskningen af institutionsbegrebet har besværliggjort muligheds for at finde en “backstage” til rollen som professionel pædagog.
Hur foretager videre en governmentality-analyse over hvordan en “frihedsdiskurs” omkring handicappede medfører nye og mere skjulte magtformer i specialpædagogisk arbejde.

Kasper Kofod skriver om den pædagogiske profession...

Tim Vikær Andersen

Martin Bayer

\section{Analyse}
hvorfor kommer de på arbejde → kerneopgaven: borgerne

hvordan? give en god dag, omsorg, aktiviteter → inddrage “omrogrsproblemet” \autocite[s.455ff]{hansbolKonstruktionAfProfessionel2008}
Arbejdet har værdi fordi det er samfundstjenligt — contra bauman s. 139

hinder? manglende hænder, rekruttering.

tiltagende administgrataionsbyrde \autocite[s. 16]{mik-meyerIndledningSkabeProfessionel2012}

skal også “være af et særligt stof” for at klare skærene.

Selviscenesættelse som exceptionelle --- et personligt karaktertræk
Bauman fællesskab — os der blev vs dem der skred/os med udd vs dem udem

oplever at blive anerkendt som eksperter i borgerne

dokumentation - stor forskel på de ufaglærte; selv opmærksomme på de mange modtagere til det skrevne; ref: \autocite{hjerrildNarViSkriver2017, andersenUndervisningInstitutionOg2019}

primært i forhold til dokumentiation, at positioneringen bliver tydelig; hvad igger fx i at 'guide'?

mere 'på' i forhold til arbejdet med ufaglærte - et nødvendig onde

klart afklaret på, at andre faggrupper har en plads

\subsection{kurser}
Kurser giver meget for at holde en opdateret

Neuro, GT, Autisme

Foucault! det du er nødt til at gøre er også en god ide - din gode ide!

Giver sammenhold og fælles retning, om ikke andet

\subsection{utilstrækkelighed}
lykkedes et tiltag ikke, er det ikke et personligt nederlag (men måske et professionelt).

Borgeren ville ikke!

\section{Konklusion}

\section{Perspektivering}
binder an til den omrsogrsdiskurs i praksis, som beskrives i \citetitle{nielsenAttraktivPaPapiret2017}. Det er det centrale for mine informantger
