\section{Indledning}

Fra en hverdagsopfattelse af professioner er velfærdsprofessionerne — herunder socialpædagoger — tilsyneladéndee udmærkede professioner. Man har tilegnet sig kompetencer, og lever af at benytte disse. \autocite[ss. 443-444]{frederiksenVelfaerdsprofessionerMellemOmsorg2017}.
Men, som \citeauthor{frederiksenVelfaerdsprofessionerMellemOmsorg2017} beskriver, er denne gruppe “professionelle” ikke helt sammenlignelige med de mere idealtypiske professioner som læger, advokater eller ingeniører.
\todo{dette er meget ufærdigt}

Lønnen er for eksempel ikke særlig høj

Der er en stor del ufaglært arbejdskraft der udfører tilsyneladende de samme opgaver \todo{lukning - frederiksen s 450-ish}

Der er begrænset autonomi


Profession og faglighed er et tilbagevendende emne i Socialpædagogen\autocite[fx]{petersenHvadSigerEksperten2019), hvor der eftersøges en social- eller specialpædagogisk “faglighed” og “professionalisme”.
Hvad er det for eksempel en pædagog på et specialpædagogisk opholdssted kan, som en ergoterapeut, en sosu-assisstent eller en socialrådgiver ikke kan? Ja, hvordan skiller en socialpædagog sig ud fra de mange ufaglærte medarbejdere på specialområdet?\todo{Hvor finder jeg de konkrete tal henne mon?}
Der synes alligevel at være bred enighed (blandt pædagogerne i hvert fald) om at de kan noget særligt og værdifuldt\todo{ref i socialpædagogen}.

Hertil kommer, hvad \citeauthor{frederiksenVelfaerdsprofessionerMellemOmsorg2017} kalder \textit{omsorgsproblemet} (\citedate[s. 455ff. ]{frederiksenVelfaerdsprofessionerMellemOmsorg2017}.
Det, at omsorgen professionaliseres, og flyttes fra familien til samfundet, medfører en særlig dualitet i arbejdet og professionsidentiteten, hvor man skal navigere mellem omsorg og kontrol \autocite[s. 461]{frederiksenVelfaerdsprofessionerMellemOmsorg2017}.
\todo{etik, normativitet, ligebehandling etc)

Molander og Terum beskriver professioner som indeholdende performative og organisatoriske aspekter.
Indenfor de performative aspekter finder man normer for, hvad der er gyldig viden, og hvad der er formålstjenlig praksis.
Her er deres skøn - et andet performativt aspekt ved professioner - bundet af udvendige praksisforskrifter og ressourcemæssige hensyn.

På det specialpædagogiske område kunne dette for eksempel være:
\begin{itemize}
  \item
    Der er handleanvisninger fra pædagogiske metoder
  \item
    Den henvisende myndighed søger handling og opfølgning — gerne med et økonomisk perspektiv
  \item
    Der er personlige og økonomiske værger, ved beslutninger om fx økonomi og sundhed
  \item
    De pårørende “flytter med” forholdsvis ofte, og har også sine egne holdninger
  \item
    Socialtilsynerne og Arbejdsmiljøtilsynet kommer med påtaler og handleanvisninger
  \item
\end{itemize}\todo{måske mere relevant lidt senere i opgaven?}

\subsection{Problemformulering}
Jeg ønsker at undersøge følgende:

Hvordan oplever og opretholder pædagoger på det specialpædagogiske område en (special)pædagogisk faglig identitet og (oplevelse af) autonomi i lyset af styring af det specialpædagogiske arbejde?

Herunder er der en rekke underspørgsmål:
Hvordan oplever pædagoger i dag deres faglige identitet?
Hvad er deres oplevelse af professionel autonomi?
Kan de mærke en ændring over tid?

Molander og Terum taler om professionsstudier på makro- og mikronviveau\todo{udddyb dette - hvad er forskellen?}. 
Idet jeg vil vide noget om, hvad udøvende praktikere har af erfaringer, er dennne undersøgelse primært på mikroniveau.
For at bringe de udøvende pædagoger i tale har jeg arrangeret et fokusgruppeinterview, med tre erfarne pædagoger.
erving goffman beskriver, hvordan de ansatte på institutionen danner en særlig kultur og 

For at perspektivere til makroniveau vil jeg beskrive, de konkrete styringsmekanismer praktikerene oplever, har betydning for deres specialpædagogiske arbejde.
Her akter jeg at støtte mig op ad Jennifer O'Day, og hendes accountability-tilgang.
Spørgsmålet om “hvem svarer til hvem, for hvad” på det specialpædagogiske område tænker jeg, kan give en dybere forståelse for, hvilke kræfter der trækker i den udøvende (special)pædagog.
En yderligere perspektivering til det omkringliggende samfund 
vil tage udgangspunkt i Campbells kognitive/normative — forgunds/baggrundsmatrix.\todo{her bliver det måske lidt meget forskellig teori på makroniveau?)


