\section{Indledning}

I følge \citeauthor{hansbolKonstruktionAfProfessionel2008} er der en gruppe erhverv, der i stigende grad ønsker anerkendelse som professioner i det 21. århundrede — herunder pædagogerne.
Spørgsmålet omkring, “hvori den professionelle identitet består” blandt de erhvervsgrupper, der primært arbejder i den offentlige sektor, skriver \citeauthor{hansbolKonstruktionAfProfessionel2008} ind i en definitionskamp omkring den offentlige sektors formål i overgagnen fra velfærdsstaten til velfærdssamfundet.
\autocite[s. 19]{hansbolKonstruktionAfProfessionel2008}.

Begrebet professionsidentitet har en særlig betydning i det moderne samfund, hvor der i stigende grad forventes, at kunne anvende sin egen personlighed i opfyldelsen af de krav, der stilles af professionen \autocite{hansbolKonstruktionAfProfessionel2008}. I følge \citeauthor{mik-meyerIndledningSkabeProfessionel2012} ender den professionelle med, at se sig selv og sit arbejde subjektivt, hvori man finder professionsidentiteten \autocite[s. 458]{mik-meyerIndledningSkabeProfessionel2012}.

\subsection{Problemformulering}

Hvordan arter det sig, at der kræves at udvikle en “individuel professionsstil” som (special)pædagog, samtidig med, at man (formodentlig) ønsker, at en personalegruppe kollektivt trækker i samme retning?

Herunder er der en rekke underspørgsmål:
Hvordan oplever pædagoger i dag deres faglige særegenhed?
Hvordan forholder de sig til de mange andre erhvervsgrupper og fagligheder de er omgivet af?
Hvad oplever de giver grundlag for fælles forståelse i personalegruppen?

\subsection{Metodiske overvejelser}

Jeg vil gerne undersøge, hvordan pædagogiske praktikere forholder sig til deres professionsidentitet.
For at bringe de udøvende pædagoger i tale har jeg arrangeret et fokusgruppeinterview med to erfarne pædagoger, der arbejder på et botilbud for voksne med kognitive funktionsnedsættelser.

\citeauthor{molanderProfesjonsstudierIntroduksjon2008} taler om, at  professionsstudier kan foretages på makro- og mikroniveau.
Idet jeg vil vide noget om, hvordan individuelle praktikere forstår og forvalter deres professionsidentitet, er denne undersøgelse primært på mikroniveau (\citeyear[][s.24]{molanderProfesjonsstudierIntroduksjon2008}.
Dog påpeger \citeauthor{molanderProfesjonsstudierIntroduksjon2008} at en mikronanalyse også vil skulle referere til de kollektive forståelser og forventninger der kendetegner makroniveauet \autocite[s. 24]{molanderProfesjonsstudierIntroduksjon2008}.
Her akter jeg at støtte mig op ad et \textit{accountability}-politisk begrebsapparat, der lægger op til, en drøftelse af “hvem, der står til ansvar til hvem, for hvad, med hvilke konsekvenser”, for at give en dybere forståelse for, hvilke kræfter der trækker i den udøvende (special)pædagog.

Min analyse vil tage udgangspunkt i Foucault, og hans beskrivelser af sociale (selv}styringsmekanismer. I tråd med den subjektiverende professionsidentitet, kan man så se skjulte magtstrukturer i, hvordan man opretholder en personlig identitet som pædagog?

\subsection{Andet arbejde på området}

Tim Vikær Andersen

\citeauthor{hurFrigorelsensMagt2015} beskriver i \citetitle{hurFrigorelsensMagt2015}, hvordan... 
Med afsæt i Goffmann og Foucault beskrives...
