\section{Indledning}

I følge \citeauthor{hansbolKonstruktionAfProfessionel2008} er der en gruppe erhverv, der i stigende grad søger og kæmper for anerkendelse som professioner i det 21. århundrede — herunder pædagogerne (\citeyear{hansbolKonstruktionAfProfessionel2008}, s. 19).
For pædagogerne har denne professionskamp været lang, og præget af diffuse krav, både til arbejdets udførelse og uddannelsens karakter \autocite[ss. 48-51]{kofodBornepolitikkenOgUdviklingen2007}.
I lighed med andre “relationsprofessioner” \autocite{moosRelationsprofessionerLaererePaedagoger2008}, “velfærdsprofessioner” \autocite{frederiksenVelfaerdsprofessionerMellemOmsorg2017} eller sågar “semiprofessioner”\autocite[s. 54]{kofodBornepolitikkenOgUdviklingen2007}, kommer pædagogerne til kort i forhold til traditionelle minimumskrav for professioner:
Uddannelse til professionen foregår ikke i en videnskabelig institution; og der er heller ikke et anerkendt monopol på, at kun pædagoger kan levere pædagogisk arbejde og pædagogiske ydelser \autocite[s.53]{kofodBornepolitikkenOgUdviklingen2007}.
På det specialiserede voksenområde udgør pædagogerne kun omkring en tredjedel af arbejdsstyrken \autocite[ss. 8-9]{kommunerneslandsforeningFaktaOmKommunernes2019}.
Hertil kommer de andre faggrupper — ergoterapeuter, sosu-hjælpere og -assistenter, der er ansat på dette område, samt en meget stor gruppe af ufaglært arbejdskraft \footnote{Den præcise andel fremgår ikke af rapporten - der henvises til en gruppe af “øvrige” på omkring halvdelen \autocite[s. 8]{kommunerneslandsforeningFaktaOmKommunernes2019}.}.

Spørgsmålet omkring, “hvori den professionelle identitet består” blandt de erhvervsgrupper, der primært arbejder i den offentlige sektor, skriver \citeauthor{hansbolKonstruktionAfProfessionel2008} ind i en definitionskamp omkring den offentlige sektors formål i overgagnen fra velfærdsstaten til velfærdssamfundet \autocite[s. 19]{hansbolKonstruktionAfProfessionel2008}.

Begrebet professionsidentitet har en særlig betydning i det moderne samfund, hvor der i stigende grad forventes, at kunne anvende sin egen personlighed i opfyldelsen af de krav, der stilles af professionen \autocite{hansbolKonstruktionAfProfessionel2008}
Ifølge \citeauthor{mik-meyerIndledningSkabeProfessionel2012} ender den professionelle med, at se sig selv og sit arbejde subjektivt, hvori man finder professionsidentiteten \autocite[s. 458]{mik-meyerIndledningSkabeProfessionel2012}.

Den enkelte pædagog, som repræsentant for sin profession, forventes at forholde sig til faglige traditioner; men også forholde sig til, at disse til stadighed er mere omskiftelige og mangfoldige \autocite[s.33]{hansbolKonstruktionAfProfessionel2008}. I følge \citeauthor{hansbolKonstruktionAfProfessionel2008}, er det netop her en “personligt forankret professionstil” er en fordel (\citeyear{hansbolKonstruktionAfProfessionel2008}, s. 33).

\citeauthor{kofodOrganisationOgLedelse2016} tager afsæt i \citeauthor{baumanLiquidModernity2000} i en beskrivelse af pædagogiske organisationer. 
Et af \citeauthor{baumanLiquidModernity2000}s pointer i \citetitle{baumanLiquidModernity2000} er, at arbejdsmarkedet i den flydende modernitet er præget af stor usikkerhed, der stiller store krav til medarbejdernes fleksibilitet \autocite[s. 147; 151]{baumanLiquidModernity2000}.
Heller ikke (special)pædagogiske arbejdspladser er immune overfor den flydende modernitets usikkerheder.
Både økonomiske konjuktursvingninger og en reduceret loyalitet til den enkelte arbejdsplads, medfører fx, at der er svagere tilknytning mellem medarbejder og institution, sammen med krav om evig omstillingsevne \autocite[s. 166f]{kofodOrganisationOgLedelse2016}.

Men hvordan tager alt dette sig ud i praksis?

\section{Problemformulering}
Jeg vil gerne undersøge, hvordan pædagogiske praktikere forholder sig til deres professionsidentitet.
Herunder er der en række underspørgsmål:

Hvordan oplever pædagoger i dag deres faglige særegenhed?

Hvordan forholder de sig til de mange andre erhvervsgrupper og fagligheder de er omgivet af?

Hvad med de andre interessenter, der forsøger at definere og give formål til pædagogers praksisforståelse?

Udtrykkes der en professionsidentitet, der er forankret i og udgør en del af de udøvende pædagogers personlighed og selvopfattelse, som beskrevet ovenfor?
I så tilfælde, hvordan?

Oplever de, noget der giver grundlag for fællesskab på arbejdspladsen?

\subsection{(Special)pædagoger som profession}
I en hverdagsopfattelse af professioner er velfærdsprofessionerne — herunder socialpædagoger — tilsyneladende udmærkede professioner.
Man har fx tilegnet sig (særlige) kompetencer, og lever af at benytte disse, på en standariseret facon. \autocite[ss. 443-445]{frederiksenVelfaerdsprofessionerMellemOmsorg2017}.

Sådan en hverdagsopfattelse af professioner er dog mangelfuld, som \citeauthor{frederiksenVelfaerdsprofessionerMellemOmsorg2017} beskriver. Den siger fx ikke noget om hvor standarden for arbejdets udførelse opstår; eller hvad “optagelseskriterierne” er \autocite[s. 445]{frederiksenVelfaerdsprofessionerMellemOmsorg2017}.
For en mere præcis beskrivelse af professioner, kan man henvende sig til professionssociologien. I følge \citeauthor{frederiksenVelfaerdsprofessionerMellemOmsorg2017}, har professionsforskningen samlet sig omkring tre retninger \autocite[s. 445]{frederiksenVelfaerdsprofessionerMellemOmsorg2017}:
\begin{itemize}
  \item
    taksonomiske professionsbegreber
  \item
    funktionalistisk professionsforståelse
  \item
    kynisk professionsforståelse
\end{itemize}

Jeg vil i det følgende gennemgå disse retninger, som beskrevet af \citeauthor{frederiksenVelfaerdsprofessionerMellemOmsorg2017}.
I denne gennemgang vil der også fremgå hvordan pædagogisk arbejde, i lighed med de andre velfærdsprofessioner, ikke helt når op til at være fuldbyrdede professioner indenfor disse tre paradigmer.

\subsubsection{Træk ved professioner}
I den taksonomiske professioneforståelse forsøger man at beskrive kendetegn ved professioner, for derefter at se om en erhvervsgrupper lever op til disse træk.
Gør den det, er den for profession at regne \autocite[s.446]{frederiksenVelfaerdsprofessionerMellemOmsorg2017}.

\citeauthor{frederiksenVelfaerdsprofessionerMellemOmsorg2017} henviser her til \citeauthor{molanderProfesjonsstudierIntroduksjon2008}, der deler en professions kendetegn op i \textit{organisatoriske} og \textit{performative} kategorier (\citeyear{frederiksenVelfaerdsprofessionerMellemOmsorg2017}, s. 446).

De organisatoriske aspekter ved professioner tilsiger, at en erhvervsgruppe der har kontrol over sine arbejdsopgaver, er en profession. Disse indbefatter \autocite[s. 18ff]{molanderProfesjonsstudierIntroduksjon2008}:
\begin{itemize}
  \item
    Monopol
  \item
    Autonomi
  \item
    Politisk konstituering
  \item
    Institutionelt imperativ
  \item
    Professionel sammenslutning
\end{itemize}

De performative aspekter ved professioner omhandler hvordan en formaliseret kompetence eller kunnen kombineres med skøn, og dermed udgør praksis. De består af, at \autocite[s 19ff]{molanderProfesjonsstudierIntroduksjon2008}:

Der ydes \textit{tjenester} til \textit{klienter},  hvor der løses \textit{praktiske problemer}.
Tjenesterne er\textit{ændringsorienterede}, og der \textit{anvendes viden} sammen med \textit{skøn}.
Tjenesterne er \textit{normativt regulerede}.
Praksis er præget af \textit{usikkerhed}, der er den \textit{professionelles ansvar}.

Pædagoger i det specialiserede voksenområde opfylder mange, men ikke alle af disse kriterier.
De har, som nævnt ovenfor, ikke monopol på at udøve (special)pædagogisk arbejde.
Der er også kun en delvis autonomi over deres arbejdsområder, idet der er flere instanser, der dikterer arbejdets udførelse og formål.
Arbejdet er politisk konstitueret i kraft af \autocite{social-ogindenrigsministerietBekendtgorelseAfLov2019}, og der er et institutionelt imperativ i, at sikre at også de svageste i samfundet har et godt og værdigt liv.
Socialpædagogerne er organiseret i Socialpædagogernes Landsforening.

Tjenesten, der leveres, er støtte til daglig livsførsel for de handicappede.
Det praktiske problem, der løses, indebærer udvikling af færdigheder og muligheder for livsudfoldelse \autocite[§ 81ff]{social-ogindenrigsministerietBekendtgorelseAfLov2019}; dette indebærer en tilstandændring hos “klienten”.
Der anvendes (i et vist omfang) specialiseret viden om specifikke problemstillinger, med skøn i forhold til den individuelle borgers aktuelle problemstillinger.
Der er normer for, hvad der anses for en værdig tilværelse og det gode liv; men pædagogen skal tage ansvar for, at kunne begå fejltrin.

Skoen trykker især omkring monopol, autonomi og vidensanvendelse, hvilket \citeauthor{frederiksenVelfaerdsprofessionerMellemOmsorg2017} påpeger, er kendetegn ved alle velfærdsprofessioner.

\subsubsection{Professionernes funktion}
Ud fra dette bud på en taksonomi over professioner kommer pædagogprofessionen til kort på nogle områder.
Den taksonomiske tilgang til professionsteori kan heller ikke beskrive, \textit{hvorfor} professioner er til, og har bestemte kendetegn \autocite[s. 450]{frederiksenVelfaerdsprofessionerMellemOmsorg2017}.

\citeauthor{frederiksenVelfaerdsprofessionerMellemOmsorg2017} henviser til Parsons, og hans funktionalistiske professionsteori, for et svar på “hvorfor professioner?”.
Professionerne har en samfundstjenlig funktion: de sikrer samfundets harmoni og moralske sammenhængskraft.
Dette kræver de ovennævnte træk ved professionerne --- uden disse, kan professionerne ikke opfylde deres samfundsmæssige funktion \autocite[s. 450]{frederiksenVelfaerdsprofessionerMellemOmsorg2017}

Det er også i denne funktionopfyldelse man skal se professionernes status, i følge \citeauthor{frederiksenVelfaerdsprofessionerMellemOmsorg2017}.
Det ansvar, der ligger i at handle på samfundtes vegne og ikke efter eget forgodtbefindende, skal også aflønnes, om ikke belønnes.
Og for at sikre, at dette sker, er en særlig uddannelse og et særligt normsæt påkrævet \autocite[s. 451]{frederiksenVelfaerdsprofessionerMellemOmsorg2017}.

Dette fordrer dog, at der er en grundlæggende konsensus i, hvad “samfundets bedste” er, og denne forklaringsmodel kommer også til kort, når der kan observeres konflikt og uenighed \autocite{frederiksenVelfaerdsprofessionerMellemOmsorg2017}.

\subsubsection{Afgrænsning og social lukning}

Profession og faglighed er også et tilbagevendende emne i Socialpædagogen \autocite[fx]{petersenHvadSigerEksperten2019}, hvor der eftersøges en særlig social- eller specialpædagogisk “faglighed” og “professionalisme”.
Hvad er det en pædagog på et specialpædagogisk opholdssted kan, som en ergoterapeut, en sosu-assisstent eller en ufaglært vikar ikke kan?

Heri er en anden form for kamp.
De forskellige fagligheder i personalegruppen stiller forskellige perspektiver på hvad, der udgør “godt arbejde”, og hvordan dette kan legitimeres som gyldig og professionel praksis.
Der synes også at være bred enighed blandt pædagogerne på Socialpædagogens FaceBook-gruppe om, at de kan noget særligt og værdifuldt \autocite{petersenSlagsMenneskeligAltmuligmand2019}, der er adskilt fra de omkringgivende fagligheder.
Denne kamp udspiller sig både lokalt, på den enkelte arbejdsplads, og i det specialpædagogiske felt i almindelighed.

Dette er et grundlæggende konfliktfylt syn på professioner, der står i kontrast til den konsensusforståelse vi ser ovenfor.

Dette forklarer \citeauthor{frederiksenVelfaerdsprofessionerMellemOmsorg2017} ud fra et kynisk perspektiv på professioner.
Status og anerkendelse er noget,der erobres.
Ved at danne en eksklusiv gruppe, der har monopol på udførelsen af et arbejde, bliver professionen bedre stillet til at forsvare sin position.
Den mest anvendte strategi for at opnå social lukning er, at kræve en bestemt uddannelse for at kunne blive optaget i professionen \autocite[s. 451ff]{frederiksenVelfaerdsprofessionerMellemOmsorg2017}.

Dette er, som vist overfor, ikke entydigt lykkedes for pædagogerne.
Det er især den store andel ufaglærte, der har tilgang til det samme arbejdsmarked, der står som elefanten i rummet.

\subsection{Pædagogerne — og de andre}

Som beskrevet i indledningen, er pædagogerne ikke de eneste erhvervsgrupper, der er beskæftiget på det specialpædagogiske område.
Jeg har beskrevet, hvordan dette udfordrer monopolet på at udføre pædagogisk arbejde.
Men det gør det også nødvendigt for pædagogerne, at forholde sig til, hvordan de adskiller sig fra eksempelvis sundhedsuddannede kolleger, der udgør omkring en tiendedel af de ansatte på det specialpædagogiske område \autocite[s. 8f]{kommunerneslandsforeningFaktaOmKommunernes2019}.

\citeauthor{porsKerneloseKerneopgaverSkolen2015} beskriver, hvordan moderne velfærdsorganisationer kan ses som “potentialitetsafsøgende organisationer”, der hele tiden afsøger, hvad de forskellige professioner kan tilbyde \autocite[s 310]{porsKerneloseKerneopgaverSkolen2015}.
Med afsæt i begrebet “kerneopgave” som noget fælles tredje, udenfor de enkelte faglige perspektivers gebet, bliver det påliggende de forskellige fagligheder at byde sig til.
Samtidig skal de være åbne for, at de andre faggrupper har noget at byde ind med.
På denne måde bliver kampen for status og position gjort tandløs, og professionerne tvinges til at tage stilling til deres egen arbejdsudsførelse \autocite[s. 311 ff.]{porsKerneloseKerneopgaverSkolen2015}.

Denne evige afsøgning gør også ansvarsområdet mere diffust. En specialpædagog skal ikke blot forholde sig til den enkelte borgers umiddelbare vanskeligheder, men til dennes livskvalitet i helhed.
Dette understreges af \citeauthor{socialpaedagogerneSocialpaedagogiskeKernefaglighed2015} skrivelse om \citetitle{socialpaedagogerneSocialpaedagogiskeKernefaglighed2015}; hvor der undrestreges, at socialpædagoger er ”den helhedsorienterede specialist” ved målrettet relationsarbejde \autocite{socialpaedagogerneSocialpaedagogiskeKernefaglighed2015}.

Dette diffuse ansvarsområde stiller også krav til den professionelle pædagogs følelsesmæssige engagement. Relationsarbejdet fordrer, at pædagogen tager brug af sin personlighed for at skabe kontakt til borgeren — men på et professionelt niveau. Dette understeger den professionelle subjektivering, idet man i stigende grad forventes at bruge hele sig selv i sit arbejde \autocite[s. 71f]{mik-meyerIndledningSkabeProfessionel2012}.

\subsection{Omsorg på det åbne marked}
Det (post)moderne samfund er kendetegnet af en differentieringsproces, der medfører en stigende pluralisme i samfundet, idet et arbejdsdelt og specialiseret samfund kræver, at individerne i et samfund bliver indbyrdes forskellige \autocite[s. 32f]{hansbolKonstruktionAfProfessionel2008}.

Velfærdsprofessionerne kendetegnes ved, at yde \textit{omsorg}, hvor de overtager opgaver, der før var forbeholdt familien \autocite[s. 445]{frederiksenVelfaerdsprofessionerMellemOmsorg2017}.
De er desuden ofte ansat i det offentlige, og en stor andel af dem er kvinder.

\citeauthor{frederiksenVelfaerdsprofessionerMellemOmsorg2017} beskriver dette som velfærdsprofessionernes omsorgsproblem \autocite[s.455ff]{frederiksenVelfaerdsprofessionerMellemOmsorg2017}.
Det er tale om
\begin{quote}
  \lodts en forpligtelse til en art intim relation, der for professionernes vedkommende ikke bæres af en tilsvarende personlig relation. \autocite[s. 456]{frederiksenVelfaerdsprofessionerMellemOmsorg2017}
\end{quote}

Denne forpligtigelse er, hos \citeauthor{frederiksenVelfaerdsprofessionerMellemOmsorg2017}, en etisk fording, der rummer etiske krav.
Dette omtales som en grundholdning hos proffesionsudøvere, og omtales som et “kald” \autocite[s. 457]{frederiksenVelfaerdsprofessionerMellemOmsorg2017}; eller pædagogens særlige “helhedsorienterede menneskesyn” fremhæves \autocite{socialpaedagogerneSocialpaedagogiskeKernefaglighed2015}. 

Omsorgsperspektivet, og de affektive elementer heri, gør videre, at den professionelles livshistorie knyttes til processen i, at udvikle sin professionsidentitet.
Man skal tage sig selv — være \textit{personlig} — med ind i sit \textit{professionelle} virke.
\autocite[s. 457f]{frederiksenVelfaerdsprofessionerMellemOmsorg2017}

Samtidig er der en stigende markedsøkonomisk orientering i tråd med en tiltagende globalisering, også i det offentlige \autocite[s. 161]{kofodOrganisationOgLedelse2016}.
Dette ser man ved, en stigende deregulering og liberalisering af pædagogiske organisationer, hvor private aktører og \textit{New Public Management} i stigende grad bliver synlige \autocite[s. 161]{kofodOrganisationOgLedelse2016}.

Man ser det også i den direkte kontakt til de omsorgstrængende. Velfærdssamfundets bruger søger i dag hjælp og vejledning til sit eget selvrealiseringsprojekt; modsat velfærdsstatens klient der henvendte sig til eksperten, over i markedssamfundets kunde, der skal kunne forvente en god og effektiv service \autocite[s. 41f ]{hansbolKonstruktionAfProfessionel2008}.

\citeauthor{frederiksenVelfaerdsprofessionerMellemOmsorg2017} beskriver velfærdsprofessionerne som værende “mellem omsorg og kontrol”. 
Kontroldimensionen fremgår, idet velfærdsstatens arbejde i dag i højere grad går ud på, at understøtte en personlig udviklingsproces \autocite[s. 461]{frederiksenVelfaerdsprofessionerMellemOmsorg2017}.
Dette fremhæves også af \citeauthor{mik-meyerIndledningSkabeProfessionel2012}, der påpeger konflikter mellem et individfokuseret og systemfokuseret problembillede.
Man skal være facilitator for udvikling, ikke en ekspert der kommer med påbud \autocite[s. 74ff]{mik-meyerIndledningSkabeProfessionel2012}.
Omsorgsarbejdet får karakter af kontrol --- tager borgeren nu det ansvar for sig selv, som selvudviklingsdiskursen lægger op til \autocite[s. 461]{frederiksenVelfaerdsprofessionerMellemOmsorg2017}?


\subsection{Professionsidentitet i diffusion}
Hvordan bidrager denne hvirvelvind af en rundrejse i pædagogprofessionens forhold og tilstand til en forståelse af professionsidentiteten blandt udøvende pædagoger?

Pædagogprofessionens status er diffus, lige som “kerneopgaven” indeholder så mange betydninger, at det kan fremstå indholdsløst \todo{dette skal jeg have kilde på}; om end der er en formålsbeskrivelse i \citetitle{social-ogindenrigsministerietBekendtgorelseAfLov2019}.

De andre faggrupper, der er beskæftiget på det specialiserede voksenområde, medfører en afgrænsning og differentiering. Både for, at kunne stå på sit, og vide hvordan man er særlig --- men også for at vide, hvor man kan byde ind og hvor man kan forvente, at de andre fagligheder har noget at bidrage med.
Denne forhandling og afsøgning overfor andre faggrupper nødvendiggør en refleksion over egen praksis, og en usikkerhed i, hvad denne praksis skal indeholde i fremtiden \autocite[s 312f]{porsKerneloseKerneopgaverSkolen2015}.

Samtidig er pædagogens rolle uklar.
Er hun en coach\footnote{Metaforen er lånt fra \citeauthor{hurFrigorelsensMagt2015}, \citeyear{hurFrigorelsensMagt2015}}, der fremskynder og faciliterer?
Eller er hun en træner, der foreskriver og hjælper?

\section{Metodiske overvejelser}

\citeauthor{molanderProfesjonsstudierIntroduksjon2008} taler om, at  professionsstudier kan foretages på makro- og mikroniveau.
Idet jeg vil vide noget om, hvordan individuelle praktikere forstår og forvalter deres professionsidentitet, er denne undersøgelse primært på mikroniveau (\citeyear{molanderProfesjonsstudierIntroduksjon2008}, s. 24).
Jeg vil nu kort ridse min tilgang til at besvare min problemformulering op, efterfulgt af en gennemgang af mine metodiske valg.

For at få de udøvende pædagoger i tale har jeg arrangeret et gruppeinterview med to pædagoger, der arbejder på et botilbud for voksne med kognitive funktionsnedsættelser.
Begge pædagoger har lang arbejdserfaring.
Jeg håber på denne måde, at få et perspektiv på (special)pædagogisk faglighed og professionalisme, der er forankret i fagets traditioner.

Min analyse vil tage udgangspunkt i Foucault, og hans beskrivelser af mekanismer, der bidrager til social (selv)styring.
I tråd med den subjektiverede professionsidentitet, kan man så se skjulte magtstrukturer i, hvordan man opretholder en personlig, professionel identitet som pædagog?\todo{Mitchell Dean?}

Dette vil jeg perspektivere til \citeauthor{baumanLiquidModernity2000}, og hans begreb om nutidens flydende modernitet (\citeyear{baumanLiquidModernity2000}).
Jeg tager afsæt i \citeauthor{kofodOrganisationOgLedelse2016}, der beskriver, hvordan \citeauthor{baumanLiquidModernity2000} kan hjælpe os til at forstå, de krav, der stilles til moderne pædagogiske arbejdspladser \autocite{kofodOrganisationOgLedelse2016}.


\subsection{Den italesatte livsverden}
\citeauthor{tanggaardInterviewetSamtalenSom2015} beskriver interviewet som

\begin{quote}
\ldots en meget almindelig måde, at opnå viden om menneskers livssituation, deres meninger, holdninger og oplevelser \autocite[s. 29]{tanggaardInterviewetSamtalenSom2015}.
\end{quote}

De beskriver videre, at interviewet kan anskues som en af midlerne i den “igangværende individuelle historiefortælling” omkring “selvet og det gode liv” \autocite[s. 30]{tanggaardInterviewetSamtalenSom2015}.
Jeg ønsker netop, i forlængelse af, at livshistorien er en central del af netop velfærdsprofessionernes identitetsdannelse \autocite[s. 457ff]{frederiksenVelfaerdsprofessionerMellemOmsorg2017}, tilgang til mine informanters oplevelser af deres livsverden — deres umiddelbare oplevelser, og ikke mindst, hvordan de italesætter dem \autocite[s. 31]{tanggaardInterviewetSamtalenSom2015}.
Ud fra et socialkonstruktionistisk udgangspunkt vil jeg her se på, hvordan særlige elementer fremhæves som betydningsfulde.
Dermed håber jeg at få øje for, hvad mine informanter anser, som vigtige og meningsbærende i deres oplevede livsverden.

\subsubsection{Hvis livsverden?}
Min empiri består af et gruppeinterview med to erfarne pædagoger.
De fremstår meget homogene — begge er kvinder, begge er over halvtreds, og begge har lang erfaring i specialpædagogisk arbejde.
Dette påvirker givetvis bredden i mit empiriske materiale.
Det kunne især have været ønskeligt, at have haft nogle informanter, med en kortere tilknytning til pædagogfaget.
Desværre var der frafald ved interviewtidspunktet, der forhindrede dette.
Jeg har dermed ikke opnået, hvad \autocite{tanggaardInterviewetSamtalenSom2015} beskriver som idealet — en “mætning” af information omkring pædagogers forholden sig til deres professionsidentitet \autocite[s. 32]{tanggaardInterviewetSamtalenSom2015}.

\subsection{Individet i det postmoderne samfund}

\subsubsection{Individet som forbruger}
\citeauthor{baumanLiquidModernity2000} beskriver i sin udlægning af det postmoderne samfund en overgang fra modernitetens faste tryghed til nutidens flydende modernitet \autocite[s. 2]{baumanLiquidModernity2000}.
Samfundets institutioner mister sin autoritet, og det enkelte individ bliver moralsk ansvarligt for sig selv \autocite[s. 64ff]{baumanLiquidModernity2000}.

Individerne i den flydende modernitet er, fremfor alt, \textit{forbrugere} \autocite[s 73ff; s. 76]{baumanLiquidModernity2000}.
Gennem forbruget køber de sig til identitetsmarkører --- herunder færdigheder til forsørgelse og midler til at overbevise arbejdsgivere at man fortjener en ansættelse \autocite[s. 74]{baumanLiquidModernity2000}.
I følge Bauman er det moderne forbrug ikke funderet i hverken behovsopfyldelse eller indfrielse af lyster - men i opfyldelse af ønsket.
Det er også heri Bauman ser betingelsen for individets frihed --- friheden til “at have identitet” er bundet op i en afhængighed til shopping \autocite[s. 84]{baumanLiquidModernity2000}.

Også på arbejdsmarkedet kan det forbrugende individ shoppe rundt.
Hvis man som arbejder ikke oplever, at ens ønsker og drømme opfyldes på sin arbejdsplads, kan man opløse forholdet og opsøge et nyt, der indehar mere “attraktive” og “spændende” kvaliteter \autocite[s. 169]{kofodOrganisationOgLedelse2016}.
Bauman anfægter, at arbejde i sig selv kan udgøre en forankring for selvforståelse og identitet.
Arbejde i det flydende moderne måles på, hvorvidt det er tilfredsstillende i sig selv; ikke hvor samfundstjenligt det måtte være \autocite[s. 139; 163f]{baumanLiquidModernity2000}.
Men denne frihed medfører også usikkerhed; der i følge Bauman også virker individuerende; i og med at det bryder med den solide modernitets forestillinger om sammenhold og solidaritet \autocite[s. 148]{baumanLiquidModernity2000}.
Og denne opløsning af den gensidige afhængighed gør sig også gældende for arbejdsgiveren.
I den flydende modernitet er det meget nemmere at skifte en arbejder ud med en anden; eller flytte sin produktion til et sted ,med billigere arbejdskraft.\todo{kilde - bauman/kofod}

\citeauthor{baumanLiquidModernity2000} henviser til Robert Reich, der har en grov typologi over dagens arbejdsstyrke (\citeyear{baumanLiquidModernity2000}, s. 152):\todo{passer måske bedre i analysen?}

Symbolanalytikerne; der opfinder og formidler;

Dem, der står for arbejdsstyrkens reproduktion;

Dem, der er beskæftiget med ansigt-til-ansigt personlig service;

Og “rutinearbejderne”; dem, der gør 'alt det andet'

Denne typologi tages også op af \citeauthor{kofodOrganisationOgLedelse2016}, der ser på Bauman i forhold til pædagogisk ledelse.
I hans udlægning lægges pædagogerne ind i gruppe 2 - “velfærdsstatens funktionærer”.
acMen samtidig lægges den sociale sektors ansatte i den tredje gruppe \autocite[s. 166]{kofodOrganisationOgLedelse2016}.
For mig at se understreger dette specialpædagogikkens identitetsproblem.
Man ivaretager og opretholder samfundets værdier på vegne af velfærdsstaten; men er til tider svære at skille fra, de ufaglærte rutinearbejdere, der også udgør en stol del af arbejdsstyrken på de specialpædagogiske arbejdspladser. 

\subsubsection{Individuerende selvstyre}\todo{mere beskrivende overskrift?}
\todo{er der sammenhæng i det her?}
\citeauthor{foucaultOvervagningOgStraf2005} viser i \citetitle{foucaultOvervagningOgStraf2005}, hvordan magtapparaterne ændrer karakter i overgangen til det moderne samfund.
Han viser, hvordan to adskilte styringsmekanismer i den disciplinerede magt sammen producerer et individueret selv \autocite[s. 186]{foucaultOvervagningOgStraf2005}.
Magten virker nu på individniveau — den adskiller, måler, og differentierer ved hjælp af den hierarkiske overvwågning, hvor de mange holde øje med de få \autocite[s. 187ff]{foucaultOvervagningOgStraf2005}.
Samtidig virker den normaliserende, hvor sociale sanktioner udskiller og korrigerer afvigelserne \autocite[s. 194]{foucaultOvervagningOgStraf2005}.
På denne måde fremhæver \citeauthor{foucaultOvervagningOgStraf2005}, hvordan “magt” ikke blot er undertrykkende og udslættende, men nu producerer subjekter, individer, der samtidig ligner hinanden \autocite[s. 198]{foucaultOvervagningOgStraf2005}.

For \citeauthor{foucaultSubjectPower1982} er det heller ikke magten, men \textit{subjektet} han har i fokus — blandt andet, hvordan mennesket subjektiverer sig selv. \autocite[s. 777f]{foucaultSubjectPower1982}.
Styring, for \citeauthor{foucaultSubjectPower1982}, er at strukturere andres handlemuligheder, og kun i den udstrækning de er frie til at vælge forskellige handlemuligheder \autocite[s. 790]{foucaultSubjectPower1982}.
Kun det frie menneske kan vælge at deltage i dette spil af magtrelationer, og udskille sig selv som subjekt \autocite[s. 789]{foucaultSubjectPower1982}.\todo{uddybe?}

Foucaults udlægning af “guvernementalitet” som en rationalisering af statens virke viser, hvordan overgangen til det moderne samfund også skaber en ny målgruppe for styring: befolkningen.
“Befolkningen” er på samme tid styringens målgruppe; men også dens formål — dennes velfærd, sikkerhed og velbefindende.
Denne guvernementalitet opererer overfor befolkningen, men også usynligt for dem \autocite[s 216ff]{foucaultGovernmentality2000}.

Besynderligt nok forsvinder individet i befolkningens masse i dette perspektiv.
Ikke dermed sagt, at befolkningen ikke består af individer — men det enkelte individ adresseres ikke direkte af guvernementaliteten.
Individet fremtræder kun, som en del af befolkningen som helhed, og har ikke nødvendigvis sammenfaldende interesser med befolkningens ønsker og behov \autocite[s. 217]{foucaultGovernmentality2000}.

\subsubsection{Forbrugende subjekter}

\citeauthor{baumanLiquidModernity2000} går i direkte dialog med styringsmekanismerne der fremlægges af \citeauthor{foucaultOvervagningOgStraf2005}.
Han beskriver det flydende moderne samfund som post-Panoptisk, idet den hierarkiske overvågningsmodel, med den evigt tilstedeværende opsynsmand; nu var opløst.
Nu brillerer opsynsmanden med sin nærmest magiske evne til at flygte fra sit ansvarsområde \autocite[s. 11]{baumanLiquidModernity2000}.
Nu er rollerne vendt på hovedet, og samfundet har nu karakter, af et Synopticon; hvor de mange holder øje med med de få \autocite[s. 85f]{baumanLiquidModernity2000}.

\citeauthor{baumanLiquidModernity2000} tegner et mere pessimistisk billede af individers selvstyring: “den frie vilje” er blot rendyrket styring til lydhørighed i forførende forkledning \autocite[s. 86]{baumanLiquidModernity2000}. 

Jeg vil i min analyse se på, hvovidt og hvordan disse magtforhold fremtræder for mine informanter på deres arbejdsplads.
\citeauthor{foucaultSubjectPower1982} omtaler institutioner som et sted, man med held undersøge sådanne mekanismer fra, idet det giver et “priviligeret observationssted” at se magtudvekslinger under ordnede forhold \autocite[s 791]{foucaultSubjectPower1982}. 
Hvordan viser disse styringsformer sig i det post-Panopiske samfund, hvor “frihed”, i følge \citeauthor{baumanLiquidModernity2000} er frihed til, at forbruge?
Hvordan subjektiveres individerne — og subjektriverer individerne sig selv — ind i personalegruppens befolkning? 

\section{Andet arbejde på området}
Det, at beskæftige mig med velfærdsprofessioner i almindelighed eller pædagogprofessionen i særdeleshed er ikke et voldsomt originalt foretagende.
Et lille uddrag af nyligt foretaget arbejde på området følger.

\citeauthor{nielsenAttraktivPaPapiret2017} har undersøgt hvordan mødet med den pædagogiske praksis er for nyuddannede pædagoger i \citetitle{nielsenAttraktivPaPapiret2017}.
De beskriver noget brat overgang og uoverensstemmmelse mellem en tiltagende akademiseret pædagoguddannelse og et relationelt forankret praksisfelt, der omtales som et “praksischok” \autocite{nielsenAttraktivPaPapiret2017}.

I \citetitle{dreyerespersenBekymrendeIdentiteterAnbragte2010} beskriver \citeauthor{dreyerespersenBekymrendeIdentiteterAnbragte2010}, med henvisning til Goffmanns institutionsanalyse, hvordan en oplevelse af særlige arbejdsmiljømæssige vilkår giver anledning til en speciel form for sammenhold og modstandsstrategier i en personalegruppe \autocite{dreyerespersenBekymrendeIdentiteterAnbragte2010}.

\citeauthor{hurFrigorelsensMagt2015} beskriver i \citetitle{hurFrigorelsensMagt2015}, hvordan udviklingen i handicapsynet og handicappolitiken har gjort grænserne for den specialpædagogiske faglighed mere diffus på nogle områder, samtidig med, at den introducerer nye styringsmekanismer overfor de handicappede \autocite{hurFrigorelsensMagt2015}.
Med afsæt i Goffmanns teatermetafor beskriver hun, hvordan udviskningen af institutionsbegrebet har besværliggjort muligheds for at finde en “backstage” til rollen som professionel pædagog.
Hur foretager videre en governmentality-analyse over hvordan en “frihedsdiskurs” omkring handicappede medfører nye og mere skjulte magtformer i specialpædagogisk arbejde.

I \citetitle{meyer-johansenFagligeOrienteringerSocialspecialpaedagogisk2018} undersøger \citeauthor{meyer-johansenFagligeOrienteringerSocialspecialpaedagogisk2018} om der er fællestræk over det brede social- og specialpædagogiske område.
Der afspejles to idealtypiske positioner: En ekspertposition, der både begrunder og legitimerer praksis, og en der vægter en stor relationel viden og forståelse af “personen” over “klienten” \autocite{meyer-johansenFagligeOrienteringerSocialspecialpaedagogisk2018}.

Mit bidrag er et blik for, hvad og hvordan erfarne pædagoger taler om sit arbejde — og, ikke mindst, taler sig selv ind i sit arbejde.\todo{blech}

\section{Analyse}
hvorfor kommer de på arbejde → kerneopgaven: borgerne

hvordan? give en god dag, omsorg, aktiviteter → inddrage “omrogrsproblemet” \autocite[s.455ff]{hansbolKonstruktionAfProfessionel2008}
Arbejdet har værdi fordi det er samfundstjenligt — contra bauman s. 139

hinder? manglende hænder, rekruttering.

tiltagende administrationsbyrde \autocite[s. 16]{mik-meyerIndledningSkabeProfessionel2012}

skal også “være af et særligt stof” for at klare skærene.

Selviscenesættelse som exceptionelle --- et personligt karaktertræk
Bauman fællesskab — os der blev vs dem der skred/os med udd vs dem uden

oplever at blive anerkendt som eksperter i borgerne

dokumentation - stor forskel på de ufaglærte; selv opmærksomme på de mange modtagere til det skrevne; ref: \autocite{hjerrildNarViSkriver2017, andersenUndervisningInstitutionOg2019}

primært i forhold til dokumentation, at positioneringen bliver tydelig; hvad ligger fx i at 'guide'?

mere 'på' i forhold til arbejdet med ufaglærte - et nødvendig onde

klart afklaret på, at andre faggrupper har en plads

\subsection{kurser}
Kurser giver meget for at holde en opdateret

Neuro, GT, Autisme

Foucault! det du er nødt til at gøre er også en god ide - din gode ide!

Giver sammenhold og fælles retning, om ikke andet

\subsection{utilstrækkelighed}
lykkedes et tiltag ikke, er det ikke et personligt nederlag (men måske et professionelt).

Borgeren ville ikke!

\section{Konklusion}

\section{Perspektivering}
binder an til den omsorgsdiskurs i praksis, som beskrives i \citetitle{nielsenAttraktivPaPapiret2017}.
Det er det centrale for mine informanter, at kunne gøre noget for en anden — helt konkret.

Hur viser hvordan Foucault er anvendelig til, at uddybe konsekvenserne af de seneste årtiers frihedsdiskurs for de handicappede, der har behov for (og får) en træner, hvor der er lagt op til en coach \autocite{hurFrigorelsensMagt2015}.
«Mit bidrag er et blik på pædagogernes gøren i praksis.

