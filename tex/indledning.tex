\section{Indledning}

I følge \citeauthor{hansbolKonstruktionAfProfessionel2008} er der en gruppe erhverv, der i stigende grad ønsker anerkendelse som professioner i det 21. århundrede — herunder pædagogerne.
Spørgsmålet omkring, “hvori den professionelle identitet består” blandt de erhvervsgrupper, der primært arbejder i den offentlige sektor, skriver \citeauthor{hansbolKonstruktionAfProfessionel2008} ind i en definitionskamp omkring den offentlige sektors formål i overgagnen fra velfærdsstaten til velfærdssamfundet.
\autocite[s. 19]{hansbolKonstruktionAfProfessionel2008}.

Profession og faglighed er også et tilbagevendende emne i Socialpædagogen \autocite[fx]{petersenHvadSigerEksperten2019}, hvor der eftersøges en social- eller specialpædagogisk “faglighed” og “professionalisme”.
Hvad er det for eksempel en pædagog på et specialpædagogisk opholdssted kan, som en ergoterapeut, en sosu-assisstent eller en socialrådgiver ikke kan? Der synes alligevel at være bred enighed (blandt pædagogerne i hvert fald) om at de kan noget særligt og værdifuldt\todo{ref i socialpædagogen}.

Jeg vil gerne grave lidt dybere i, hvordan udøvende (social)pædagoger ser sig selv som “noget særligt”, der skiller sig ud fra andre faggrupper der arbejder med mennesker med handicap.

\subsection{Andet arbejde på området}

Tim Vikær Andersen

Kasper Kofod

\citeauthor{hurFrigorelsensMagt2015} beskriver i \citetitle{hurFrigorelsensMagt2015), hvordan... 
Med afsæt i Goffmann og Foucault beskrives...

\subsection{Problemformulering}

Jeg vil forsøge at give et bud på, hvordan en professionel identitet viser sig hos og for pædagoger på et botilbud for voksne med kognitive funktionsnedsættelser.

Herunder er der en rekke underspørgsmål:
Hvordan oplever pædagoger i dag deres faglige identitet?
Hvad er deres oplevelse af professionel autonomi?
Kan de mærke en ændring over tid?

\subsection{Metodiske overvejelser}
For at bringe de udøvende pædagoger i tale har jeg arrangeret et fokusgruppeinterview, med to erfarne pædagoger.

\citeauthor{molanderProfesjonsstudierIntroduksjon2008} taler om, at  professionsstudier kan foretages på makro- og mikroniveau.
Idet jeg vil vide noget om, hvordan individuelle praktikere forstår og forvalter deres professionsidentitet, er denne undersøgelse primært på mikroniveau.
Dog påpeger \citeauthor{molanderProfesjonsstudierIntroduksjon2008} at en mikronanalyse også vil skulle referere til de kollektive forståelser og forventninger der kendetegner makroniveauet \autocite[s. 24]{molanderProfesjonsstudierIntroduksjon2008}.
Her akter jeg at støtte mig op ad \citeauthor{odayComplexityAccountabilitySchool2002}, og hendes beskrivelser af accountability.
En drøftelse af “hvem, der svarer til hvem, for hvad” på det specialpædagogiske område tænker jeg, kan give en dybere forståelse for, hvilke kræfter der trækker i den udøvende (special)pædagog.

