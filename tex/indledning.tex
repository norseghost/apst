\section{Indledning}

I følge \citeauthor{hansbolKonstruktionAfProfessionel2008} er der en gruppe erhverv, der i stigende grad søger og kæmper for anerkendelse som professioner i det 21. århundrede — herunder pædagogerne (\citeyear{hansbolKonstruktionAfProfessionel2008}, s. 19).
For pædagogerne har denne professionskamp været lang, og præget af diffuse krav, både til arbejdets udførelse og uddannelsens karakter \autocite[ss. 48-51] {kofodBornepolitikkenOgUdviklingen2007}.
I lighed med andre “relationsprofessioner”\todo{kilde}, “velfærdsprofessioner”\todo{kilde} eller sågar “semiprofessioner”\todo{kilde}, kommer pædagogerne til kort i forhold til traditionelle minimumskrav for professioner:
Uddannelse til professionen foregår ikke i en videnskabelig institution; og der er heller ikke et anerkendt monopol på, at kun pædagoger kan levere pædagogisk arbejde og pædagogiske ydelser \autocite[s.53]{kofodBornepolitikkenOgUdviklingen2007}.
På det specialserede voksenområde udføres omkring halvdelen af arbejdet af ufaglært arbejdskraft.\todo{kilde}
Hertil kommer de andre faggrupper — ergoterapeuter, sosu-hjælpere oge -assistenter, der er ansat på dette område.

Spørgsmålet omkring, “hvori den professionelle identitet består” blandt de erhvervsgrupper, der primært arbejder i den offentlige sektor, skriver \citeauthor{hansbolKonstruktionAfProfessionel2008} ind i en definitionskamp omkring den offentlige sektors formål i overgagnen fra velfærdsstaten til velfærdssamfundet.
\autocite[s. 19]{hansbolKonstruktionAfProfessionel2008}.

Begrebet professionsidentitet har en særlig betydning i det moderne samfund, hvor der i stigende grad forventes, at kunne anvende sin egen personlighed i opfyldelsen af de krav, der stilles af professionen \autocite{hansbolKonstruktionAfProfessionel2008}. I følge \citeauthor{mik-meyerIndledningSkabeProfessionel2012} ender den professionelle med, at se sig selv og sit arbejde subjektivt, hvori man finder professionsidentiteten \autocite[s. 458]{mik-meyerIndledningSkabeProfessionel2012}.

Den enkelte pædagog, som repræsentant for sin profession, forventes at forholde sig til faglige traditioner; men også forholde sig til, at disse er til stadighed mere omskiftelige og mangfoldige \autocite{hansbolKonstruktionAfProfessionel2008}. I følge \citeauthor{hansbolKonstruktionAfProfessionel2008}, er det netop her en “personligt forankret profeessionstil” er en fordel (\citeyear{hansbolKonstruktionAfProfessionel2008}, s. 33).

\todo{noget med en personalegruppe}

Men hvordan tager det sig ud i praksis?

\section{Problemformulering}
Jeg vil gerne undersøge, hvordan pædagogiske praktikere forholder sig til deres professionsidentitet.
Herunder er der en række underspørgsmål:\todo{måske flere efterhånden som analysen skrider frem?}

Hvordan oplever pædagoger i dag deres faglige særegenhed?

Hvordan forholder de sig til de mange andre erhvervsgrupper og fagligheder de er omgivet af?

Hvad med de andre interessenter, der forsøger at definere og give formål til pædagogers praksisforståelse?

Udtrykkes der en professionsidentitet, der er forankret i og udgør en del af de udøvende pædagogers personlighed og selvopfattelse, som der står ovenfor?
I så tilfælde, hvordan?

Hvad oplever de giver grundlag for fælles forståelse i personalegruppen?\todo{måske ikke explicit i PF?}

\subsection{Pædagogers profession og faglighed}
Profession og faglighed er et tilbagevendende emne i Socialpædagogen \autocite[fx]{petersenHvadSigerEksperten2019}, hvor der eftersøges en særlig social- eller specialpædagogisk “faglighed” og “professionalisme”.
Hvad er det en pædagog på et specialpædagogisk opholdssted kan, som en ergoterapeut, en sosu-assisstent eller en ufaglært vikar ikke kan?

Heri er en anden form for kamp, lokalt på — i dette tilfælde — den specialpædaogiske institution.
De forskellige fagligheder i personalegruppen stiller forskellige perspektiver på hvad, der udgør “godt arbejde”, og hvordan dette kan legitimeres som gyldig og professionel praksis.Der synes også at være bred enighed blandt pædagogerne på Socialpædagogens facebook-gruppe om, at de kan noget særligt og værdifuldt \autocite{petersenSlagsMenneskeligAltmuligmand2019}, der er adskilt fra de omkringgivende fagligheder.

\subsection{Afgrænsning og social lukning}
Dette er et grundlæggende konfliktfylt syn på professioner \todo{give et bourdieusk eller neo weberiansk perspektiv på denne konfliktualitet}

Noget med tværfagligt samarbejde - og den “kerneløse kerneopgave” i en “potentialitetsafsøgende institution”\todo{undersøg formulering} \autocite{mik-meyerIndledningSkabeProfessionel2012}

Hvilket udvider den professionelle subjektivering til også at se andre professioner som noget instrumentelt? måske?

\subsection{Andre interessenter}
\todo{this heading sux}
Det (special)pædagogiske arbejdsområde har også det til følles med andre  relationsprofessioner, at de ikke har fuld kontrol over deres arbejdes form og indhold

noget med markedsstat og pædagogen som forretnigsmand?

hvem er egentlig 'kunden' på det specialpædagogiske område
\section{Metodiske overvejelser}

For at få de udøvende pædagoger i tale har jeg arrangeret et fokusgruppeinterview med pædagoger, der arbejder på et botilbud for voksne med kognitive funktionsnedsættelser.
Min fokusgruppe består af to pædagoger, begge med lang arbejdserfaring.
Mit empiriske materiale er dermed et perspektiv på (special)pædagogisk faglighed og professionalisme, der har en lang forankring i fagets traditioner.\todo{uddybe, hvilken betydning dette har for min undersøgelse}

Min analyse vil tage udgangspunkt i Foucault, og hans beskrivelser af mekanismer, der bidrager til social (selv)styring. I tråd med den subjektivere(n)de professionsidentitet, kan man så se skjulte magtstrukturer i, hvordan man opretholder en personlig, professionel identitet som pædagog?

\citeauthor{molanderProfesjonsstudierIntroduksjon2008} taler om, at  professionsstudier kan foretages på makro- og mikroniveau.
Idet jeg vil vide noget om, hvordan individuelle praktikere forstår og forvalter deres professionsidentitet, er denne undersøgelse primært på mikroniveau (\citeyear{molanderProfesjonsstudierIntroduksjon2008}, s. 24).
Dog påpeger \citeauthor{molanderProfesjonsstudierIntroduksjon2008} at en mikronanalyse også vil skulle referere til de kollektive forståelser og forventninger der kendetegner makroniveauet \autocite[s. 24]{molanderProfesjonsstudierIntroduksjon2008}.
Her vil jeg udforske den førnævnte definitionskamp omkring velfærdsstatens rolle i det moderne samfund.\todo{bedre overgang til acct}
Her akter jeg at støtte mig op ad et \textit{accountability}-politisk begrebsapparat, der lægger op til, en drøftelse af “hvem, der står til ansvar til hvem, for hvad, med hvilke konsekvenser”, for at give en dybere forståelse for, hvilke kræfter der trækker i den udøvende (special)pædagog.\todo{relevans til problemformulering?}

\subsection{Andet arbejde på området}
\citeauthor{nielsenAttraktivPaPapiret2017} har undersøgt hvordan mødet med den pædagogiske praksis er for nyuddannede pædagoger. De beskriver noget brat overgang og uoverensstemmmelse mellem en tiltagende akademiseret pædagoguddannelse og et relationelt forankret praksisfelt \autocite{nielsenAttraktivPaPapiret2017}.
I \citetitle{dreyerespersenBekymrendeIdentiteterAnbragte2010} beskriver \citeauthor{dreyerespersenBekymrendeIdentiteterAnbragte2010} med henvisning til Goffmanns institutionsanalyse hvordan en oplevelse af særlige arbejdsmiljømæssige vilkår giver anledning til en speciel form for sammenhold og modstandsstrategier i en personalegruppe \autocite{dreyerespersenBekymrendeIdentiteterAnbragte2010}.

\citeauthor{hurFrigorelsensMagt2015} beskriver i \citetitle{hurFrigorelsensMagt2015}, hvordan udviklingen i handicapsynet og handicappolitiken har gjort grænserne for den specialpædagogiske faglighed mere diffus på nogle områder.
Med afsæt i Goffmanns teatermetafor beskriver hun, hvordan udviskningen af institutionsbegrebet har besværliggjort mulighedn for at finde en “backstage” til rollen som professionel pæadagog.

Foucault beskrives\todo{hvordan bruger Hur Foucault?}


Kasper Kofod skriver om den pædagogiske profession...

Tim Vikær Andersen

Martin Bayer

\section{Analyse}
hvorfor kommer de på arbejde → kerneopgaven: borgerne

hvordan? give en god dag, omsorg, aktiviteter → inddrage “omrogrsproblemet” \autocite[s.455ff]{hansbolKonstruktionAfProfessionel2008}

hinder? manglende hænder, rekruttering.

skal også “være af et særligt stof” for at klare skærene.

Selviscenesættelse som exceptionelle --- et personligt karaktertræk

oplever at blive anerkendt som eksperter i borgerne

dokumentation - stor forskel på de ufaglærte; selv opmærksomme på de mange modtagere til det skrevne; ref: \autocite{hjerrildNarViSkriver2017, andersenUndervisningInstitutionOg2019}

primært i forhold til dokumentiation, at positioneringen bliver tydelig; hvad igger fx i at 'guide'?

mere 'på' i forhold til arbejdet med ufaglærte - et nødvendig onde

klart afklaret på, at andre faggrupper har en plads

\subsection{kurser}
Kurser giver meget for at holde en opdateret

Neuro, GT, Autisme

Foucault! det du er nødt til at gøre er også en god ide - din gode ide!

Giver sammenhold og fælles retning, om ikke andet

\subsection{utilstrækkelighed}
lykkedes et tiltag ikke, er det ikke et personligt nederlag (men måske et professionelt}.

Borgeren ville ikke!

\section{Konklusion}

\section{Perspektivering}
binder an til den omrsogrsdiskurs i praksis, som beskrives i \citetitle{nielsenAttraktivPaPapiret2017}. Det er det centrale for mine informantger
