\section{Indledning}

I følge \citeauthor{hansbolKonstruktionAfProfessionel2008} er der en gruppe erhverv, der i stigende grad søger og kæmper for anerkendelse som professioner i det 21. århundrede — herunder pædagogerne (\citeyear{hansbolKonstruktionAfProfessionel2008}, s. 19).
For pædagogerne har denne professionskamp været lang, og præget af diffuse krav, både til arbejdets udførelse og uddannelsens karakter \autocite[ss. 48-51]{kofodBornepolitikkenOgUdviklingen2007}.
I lighed med andre “relationsprofessioner” \autocite{moosRelationsprofessionerLaererePaedagoger2008}, “velfærdsprofessioner” \autocite{frederiksenVelfaerdsprofessionerMellemOmsorg2017} eller sågar “semiprofessioner”\autocite[s. 54]{kofodBornepolitikkenOgUdviklingen2007}, kommer pædagogerne til kort i forhold til traditionelle minimumskrav for professioner:
Uddannelse til professionen foregår ikke i en videnskabelig institution; og der er heller ikke et anerkendt monopol på, at kun pædagoger kan levere pædagogisk arbejde og pædagogiske ydelser \autocite[s.53]{kofodBornepolitikkenOgUdviklingen2007}.
På det specialiserede voksenområde udgør pædagogerne kun omkring en tredjedel af arbejdsstyrken \autocite[ss. 8-9]{kommunerneslandsforeningFaktaOmKommunernes2019}.
Hertil kommer de andre faggrupper — ergoterapeuter, sosu-hjælpere og -assistenter, der er ansat på dette område, samt en meget stor gruppe af ufaglært arbejdskraft \footnote{Den præcise andel fremgår ikke af rapporten - der henvises til en gruppe af “øvrige” på omkring halvdelen \autocite[s. 8]{kommunerneslandsforeningFaktaOmKommunernes2019}.}.

Spørgsmålet omkring, “hvori den professionelle identitet består” blandt de erhvervsgrupper, der primært arbejder i den offentlige sektor, skriver \citeauthor{hansbolKonstruktionAfProfessionel2008} ind i en definitionskamp omkring den offentlige sektors formål i overgagnen fra velfærdsstaten til velfærdssamfundet \autocite[s. 19]{hansbolKonstruktionAfProfessionel2008}.

Begrebet professionsidentitet har en særlig betydning i det moderne samfund, hvor der i stigende grad forventes, at kunne anvende sin egen personlighed i opfyldelsen af de krav, der stilles af professionen \autocite{hansbolKonstruktionAfProfessionel2008}
Ifølge \citeauthor{mik-meyerIndledningSkabeProfessionel2012} ender den professionelle med, at se sig selv og sit arbejde subjektivt, hvori man finder professionsidentiteten \autocite[s. 458]{mik-meyerIndledningSkabeProfessionel2012}.

Den enkelte pædagog, som repræsentant for sin profession, forventes at forholde sig til faglige traditioner; men også forholde sig til, at disse til stadighed er mere omskiftelige og mangfoldige \autocite[s.33]{hansbolKonstruktionAfProfessionel2008}. I følge \citeauthor{hansbolKonstruktionAfProfessionel2008}, er det netop her en “personligt forankret professionstil” er en fordel (\citeyear{hansbolKonstruktionAfProfessionel2008}, s. 33).

\citeauthor{kofodOrganisationOgLedelse2016} tager afsæt i \citeauthor{baumanLiquidModernity2000} i en beskrivelse af pædagogiske organisationer. 
Et af \citeauthor{baumanLiquidModernity2000}s pointer i \citetitle{baumanLiquidModernity2000} er, at arbejdsmarkedet i den flydende modernitet er præget af stor usikkerhed, der stiller store krav til medarbejdernes fleksibilitet \autocite[s. 147; 151]{baumanLiquidModernity2000}.
Heller ikke (special)pædagogiske arbejdspladser er immune overfor den flydende modernitets usikkerheder.
Både økonomiske konjuktursvingninger og en reduceret loyalitet til den enkelte arbejdsplads, medfører fx, at der er svagere tilknytning mellem medarbejder og institution, sammen med krav om evig omstillingsevne \autocite[s. 166f]{kofodOrganisationOgLedelse2016}.

Men hvordan tager alt dette sig ud i praksis?

\section{Problemformulering}
Jeg vil gerne undersøge, hvordan pædagogiske praktikere forholder sig til deres professionsidentitet.
Herunder er der en række underspørgsmål:

Hvordan oplever pædagoger i dag deres faglige særegenhed?

Hvordan forholder de sig til de mange andre erhvervsgrupper og fagligheder de er omgivet af?

Hvad med de andre interessenter, der forsøger at definere og give formål til pædagogers praksisforståelse?

Udtrykkes der en professionsidentitet, der er forankret i og udgør en del af de udøvende pædagogers personlighed og selvopfattelse, som beskrevet ovenfor?
I så tilfælde, hvordan?

Oplever de, noget der giver grundlag for fællesskab på arbejdspladsen?

\subsection{(Special)pædagoger som profession}
I en hverdagsopfattelse af professioner er velfærdsprofessionerne — herunder special/socialpædagoger — tilsyneladende udmærkede professioner.
Man har fx tilegnet sig (særlige) kompetencer, og lever af at benytte disse, på en standariseret facon. \autocite[ss. 443-445]{frederiksenVelfaerdsprofessionerMellemOmsorg2017}.

Sådan en hverdagsopfattelse af professioner er dog mangelfuld, som \citeauthor{frederiksenVelfaerdsprofessionerMellemOmsorg2017} beskriver. Den siger fx ikke noget om hvor standarden for arbejdets udførelse opstår; eller hvad “optagelseskriterierne” er \autocite[s. 445]{frederiksenVelfaerdsprofessionerMellemOmsorg2017}.
For en mere præcis beskrivelse af professioner, kan man henvende sig til professionssociologien. I følge \citeauthor{frederiksenVelfaerdsprofessionerMellemOmsorg2017}, har professionsforskningen samlet sig omkring tre retninger \autocite[s. 445]{frederiksenVelfaerdsprofessionerMellemOmsorg2017}:
\begin{itemize}
  \item
    taksonomiske professionsbegreber
  \item
    funktionalistisk professionsforståelse
  \item
    kynisk professionsforståelse
\end{itemize}

Jeg vil i det følgende gennemgå disse retninger, som beskrevet af \citeauthor{frederiksenVelfaerdsprofessionerMellemOmsorg2017}.
I denne gennemgang vil der også fremgå hvordan pædagogisk arbejde, i lighed med de andre velfærdsprofessioner, ikke helt når op til at være fuldbyrdede professioner indenfor disse tre paradigmer.

\subsubsection{Træk ved professioner}
I den taksonomiske professioneforståelse forsøger man at beskrive kendetegn ved professioner, for derefter at se om en erhvervsgrupper lever op til disse træk.
Gør den det, er den for profession at regne \autocite[s.446]{frederiksenVelfaerdsprofessionerMellemOmsorg2017}.

\citeauthor{frederiksenVelfaerdsprofessionerMellemOmsorg2017} henviser her til \citeauthor{molanderProfesjonsstudierIntroduksjon2008}, der deler en professions kendetegn op i \textit{organisatoriske} og \textit{performative} kategorier (\citeyear{frederiksenVelfaerdsprofessionerMellemOmsorg2017}, s. 446).

De organisatoriske aspekter ved professioner tilsiger, at en erhvervsgruppe der har kontrol over sine arbejdsopgaver, er en profession. Disse indbefatter \autocite[s. 18ff]{molanderProfesjonsstudierIntroduksjon2008}:
\begin{itemize}
  \item
    Monopol
  \item
    Autonomi
  \item
    Politisk konstituering
  \item
    Institutionelt imperativ
  \item
    Professionel sammenslutning
\end{itemize}

De performative aspekter ved professioner omhandler hvordan en formaliseret kompetence eller kunnen kombineres med skøn, og dermed udgør praksis. De består af, at \autocite[s 19ff]{molanderProfesjonsstudierIntroduksjon2008}:

Der ydes \textit{tjenester} til \textit{klienter},  hvor der løses \textit{praktiske problemer}.
Tjenesterne er\textit{ændringsorienterede}, og der \textit{anvendes viden} sammen med \textit{skøn}.
Tjenesterne er \textit{normativt regulerede}.
Praksis er præget af \textit{usikkerhed}, der er den \textit{professionelles ansvar}.

Pædagoger i det specialiserede voksenområde opfylder mange, men ikke alle af disse kriterier.
De har, som nævnt ovenfor, ikke monopol på at udøve (special)pædagogisk arbejde.
Der er også kun en delvis autonomi over deres arbejdsområder, idet der er flere instanser, der dikterer arbejdets udførelse og formål.
Arbejdet er politisk konstitueret i kraft af \autocite{social-ogindenrigsministerietBekendtgorelseAfLov2019}, og der er et institutionelt imperativ i, at sikre at også de svageste i samfundet har et godt og værdigt liv.
Socialpædagogerne er organiseret i Socialpædagogernes Landsforening.

Tjenesten, der leveres, er støtte til daglig livsførsel for de handicappede.
Det praktiske problem, der løses, indebærer udvikling af færdigheder og muligheder for livsudfoldelse \autocite[§ 81ff]{social-ogindenrigsministerietBekendtgorelseAfLov2019}; dette indebærer en tilstandændring hos “klienten”.
Der anvendes (i et vist omfang) specialiseret viden om specifikke problemstillinger, med skøn i forhold til den individuelle borgers aktuelle problemstillinger.
Der er normer for, hvad der anses for en værdig tilværelse og det gode liv; men pædagogen skal tage ansvar for, at kunne begå fejltrin.

Skoen trykker især omkring monopol, autonomi og vidensanvendelse, hvilket \citeauthor{frederiksenVelfaerdsprofessionerMellemOmsorg2017} påpeger, er kendetegn ved alle velfærdsprofessioner.

\subsubsection{Professionernes funktion}
Ud fra dette bud på en taksonomi over professioner kommer pædagogprofessionen til kort på nogle områder.
Den taksonomiske tilgang til professionsteori kan heller ikke beskrive, \textit{hvorfor} professioner er til, og har bestemte kendetegn \autocite[s. 450]{frederiksenVelfaerdsprofessionerMellemOmsorg2017}.

\citeauthor{frederiksenVelfaerdsprofessionerMellemOmsorg2017} henviser til Parsons, og hans funktionalistiske professionsteori, for et svar på “hvorfor professioner?”.
Professionerne har en samfundstjenlig funktion: de sikrer samfundets harmoni og moralske sammenhængskraft.
Dette kræver de ovennævnte træk ved professionerne --- uden disse, kan professionerne ikke opfylde deres samfundsmæssige funktion \autocite[s. 450]{frederiksenVelfaerdsprofessionerMellemOmsorg2017}

Det er også i denne funktionopfyldelse man skal se professionernes status, i følge \citeauthor{frederiksenVelfaerdsprofessionerMellemOmsorg2017}.
Det ansvar, der ligger i at handle på samfundtes vegne og ikke efter eget forgodtbefindende, skal også aflønnes, om ikke belønnes.
Og for at sikre, at dette sker, er en særlig uddannelse og et særligt normsæt påkrævet \autocite[s. 451]{frederiksenVelfaerdsprofessionerMellemOmsorg2017}.

Dette fordrer dog, at der er en grundlæggende konsensus i, hvad “samfundets bedste” er, og denne forklaringsmodel kommer også til kort, når der kan observeres konflikt og uenighed \autocite{frederiksenVelfaerdsprofessionerMellemOmsorg2017}.

\subsubsection{Afgrænsning og social lukning}

Profession og faglighed er også et tilbagevendende emne i Socialpædagogen \autocite[fx]{petersenHvadSigerEksperten2019}, hvor der eftersøges en særlig social- eller specialpædagogisk “faglighed” og “professionalisme”.
Hvad er det en pædagog på et specialpædagogisk opholdssted kan, som en ergoterapeut, en sosu-assisstent eller en ufaglært vikar ikke kan?

Heri er en anden form for kamp.
De forskellige fagligheder i personalegruppen stiller forskellige perspektiver på hvad, der udgør “godt arbejde”, og hvordan dette kan legitimeres som gyldig og professionel praksis.
Der synes også at være bred enighed blandt pædagogerne på Socialpædagogens FaceBook-gruppe om, at de kan noget særligt og værdifuldt \autocite{petersenSlagsMenneskeligAltmuligmand2019}, der er adskilt fra de omkringgivende fagligheder.
Denne kamp udspiller sig både lokalt, på den enkelte arbejdsplads, og i det specialpædagogiske felt i almindelighed.

Dette er et grundlæggende konfliktfylt syn på professioner, der står i kontrast til den konsensusforståelse vi ser ovenfor.

Dette forklarer \citeauthor{frederiksenVelfaerdsprofessionerMellemOmsorg2017} ud fra et kynisk perspektiv på professioner.
Status og anerkendelse er noget,der erobres.
Ved at danne en eksklusiv gruppe, der har monopol på udførelsen af et arbejde, bliver professionen bedre stillet til at forsvare sin position.
Den mest anvendte strategi for at opnå social lukning er, at kræve en bestemt uddannelse for at kunne blive optaget i professionen \autocite[s. 451ff]{frederiksenVelfaerdsprofessionerMellemOmsorg2017}.

Dette er, som vist overfor, ikke entydigt lykkedes for pædagogerne.
Det er især den store andel ufaglærte, der har tilgang til det samme arbejdsmarked, der står som elefanten i rummet.

\subsection{Pædagogerne — og de andre}

Som beskrevet i indledningen, er pædagogerne ikke de eneste erhvervsgrupper, der er beskæftiget på det specialpædagogiske område.
Jeg har beskrevet, hvordan dette udfordrer monopolet på at udføre pædagogisk arbejde.
Men det gør det også nødvendigt for pædagogerne, at forholde sig til, hvordan de adskiller sig fra eksempelvis sundhedsuddannede kolleger, der udgør omkring en tiendedel af de ansatte på det specialpædagogiske område \autocite[s. 8f]{kommunerneslandsforeningFaktaOmKommunernes2019}.

\citeauthor{porsKerneloseKerneopgaverSkolen2015} beskriver, hvordan moderne velfærdsorganisationer kan ses som “potentialitetsafsøgende organisationer”, der hele tiden afsøger, hvad de forskellige professioner kan tilbyde \autocite[s 310]{porsKerneloseKerneopgaverSkolen2015}.
Med afsæt i begrebet “kerneopgave” som noget fælles tredje, udenfor de enkelte faglige perspektivers gebet, bliver det påliggende de forskellige fagligheder at byde sig til.
Samtidig skal de være åbne for, at de andre faggrupper har noget at byde ind med.
På denne måde bliver kampen for status og position gjort tandløs, og professionerne tvinges til at tage stilling til deres egen arbejdsudsførelse \autocite[s. 311 ff.]{porsKerneloseKerneopgaverSkolen2015}.

Denne evige afsøgning gør også ansvarsområdet mere diffust. En specialpædagog skal ikke blot forholde sig til den enkelte borgers umiddelbare vanskeligheder, men til dennes livskvalitet i helhed.
Dette understreges af \citeauthor{socialpaedagogerneSocialpaedagogiskeKernefaglighed2015} skrivelse om \citetitle{socialpaedagogerneSocialpaedagogiskeKernefaglighed2015}; hvor der undrestreges, at socialpædagoger er ”den helhedsorienterede specialist” ved målrettet relationsarbejde \autocite{socialpaedagogerneSocialpaedagogiskeKernefaglighed2015}.

Dette diffuse ansvarsområde stiller også krav til den professionelle pædagogs følelsesmæssige engagement. Relationsarbejdet fordrer, at pædagogen tager brug af sin personlighed for at skabe kontakt til borgeren — men på et professionelt niveau. Dette understeger den professionelle subjektivering, idet man i stigende grad forventes at bruge hele sig selv i sit arbejde \autocite[s. 71f]{mik-meyerIndledningSkabeProfessionel2012}.

\subsection{Omsorg på det åbne marked}
Det (post)moderne samfund er kendetegnet af en differentieringsproces, der medfører en stigende pluralisme i samfundet, idet et arbejdsdelt og specialiseret samfund kræver, at individerne i et samfund bliver indbyrdes forskellige \autocite[s. 32f]{hansbolKonstruktionAfProfessionel2008}.

Velfærdsprofessionerne kendetegnes ved, at yde \textit{omsorg}, hvor de overtager opgaver, der før var forbeholdt familien \autocite[s. 445]{frederiksenVelfaerdsprofessionerMellemOmsorg2017}.
De er desuden ofte ansat i det offentlige, og en stor andel af dem er kvinder.

\citeauthor{frederiksenVelfaerdsprofessionerMellemOmsorg2017} beskriver dette som velfærdsprofessionernes omsorgsproblem \autocite[s.455ff]{frederiksenVelfaerdsprofessionerMellemOmsorg2017}.
Det er tale om
\begin{quote}
  \ldots en forpligtelse til en art intim relation, der for professionernes vedkommende ikke bæres af en tilsvarende personlig relation. \autocite[s. 456]{frederiksenVelfaerdsprofessionerMellemOmsorg2017}
\end{quote}

Denne forpligtigelse er, hos \citeauthor{frederiksenVelfaerdsprofessionerMellemOmsorg2017}, en etisk fording, der rummer etiske krav.
Dette omtales som en grundholdning hos proffesionsudøvere, og omtales som et “kald” \autocite[s. 457]{frederiksenVelfaerdsprofessionerMellemOmsorg2017}; eller pædagogens særlige “helhedsorienterede menneskesyn” fremhæves \autocite{socialpaedagogerneSocialpaedagogiskeKernefaglighed2015}. 

Omsorgsperspektivet, og de affektive elementer heri, gør videre, at den professionelles livshistorie knyttes til processen i, at udvikle sin professionsidentitet.
Man skal tage sig selv — være \textit{personlig} — med ind i sit \textit{professionelle} virke.
\autocite[s. 457f]{frederiksenVelfaerdsprofessionerMellemOmsorg2017}

Samtidig er der en stigende markedsøkonomisk orientering i tråd med en tiltagende globalisering, også i det offentlige \autocite[s. 161]{kofodOrganisationOgLedelse2016}.
Dette ser man ved, en stigende deregulering og liberalisering af pædagogiske organisationer, hvor private aktører og \textit{New Public Management} i stigende grad bliver synlige \autocite[s. 161]{kofodOrganisationOgLedelse2016}.

Man ser det også i den direkte kontakt til de omsorgstrængende. Velfærdssamfundets bruger søger i dag hjælp og vejledning til sit eget selvrealiseringsprojekt; modsat velfærdsstatens klient der henvendte sig til eksperten, over i markedssamfundets kunde, der skal kunne forvente en god og effektiv service \autocite[s. 41f ]{hansbolKonstruktionAfProfessionel2008}.

\citeauthor{frederiksenVelfaerdsprofessionerMellemOmsorg2017} beskriver velfærdsprofessionerne som værende “mellem omsorg og kontrol”. 
Kontroldimensionen fremgår, idet velfærdsstatens arbejde i dag i højere grad går ud på, at understøtte en personlig udviklingsproces \autocite[s. 461]{frederiksenVelfaerdsprofessionerMellemOmsorg2017}.
Dette fremhæves også af \citeauthor{mik-meyerIndledningSkabeProfessionel2012}, der påpeger konflikter mellem et individfokuseret og systemfokuseret problembillede.
Man skal være facilitator for udvikling, ikke en ekspert der kommer med påbud \autocite[s. 74ff]{mik-meyerIndledningSkabeProfessionel2012}.
Omsorgsarbejdet får karakter af kontrol --- tager borgeren nu det ansvar for sig selv, som selvudviklingsdiskursen lægger op til \autocite[s. 461]{frederiksenVelfaerdsprofessionerMellemOmsorg2017}?


\subsection{Professionsidentitet i diffusion}
Hvordan bidrager denne hvirvelvind af en rundrejse i pædagogprofessionens forhold og tilstand til en forståelse af professionsidentiteten blandt udøvende pædagoger?

Pædagogprofessionens status er diffus, lige som “kerneopgaven” indeholder så mange betydninger, at det kan fremstå indholdsløst \autocite[s.314f]{porsKerneloseKerneopgaverSkolen2015}; om end der er en formålsbeskrivelse i \citetitle{social-ogindenrigsministerietBekendtgorelseAfLov2019}.

De andre faggrupper, der er beskæftiget på det specialiserede voksenområde, medfører en afgrænsning og differentiering. Både for, at kunne stå på sit, og vide hvordan man er særlig --- men også for at vide, hvor man kan byde ind og hvor man kan forvente, at de andre fagligheder har noget at bidrage med.
Denne forhandling og afsøgning overfor andre faggrupper nødvendiggør en refleksion over egen praksis, og en usikkerhed i, hvad denne praksis skal indeholde i fremtiden \autocite[s 312f]{porsKerneloseKerneopgaverSkolen2015}.

Samtidig er pædagogens rolle uklar.
Er hun en coach\footnote{Metaforen er lånt fra \citeauthor{hurFrigorelsensMagt2015}, \citeyear{hurFrigorelsensMagt2015}}, der fremskynder og faciliterer?
Eller er hun en træner, der foreskriver og hjælper?

\section{Metodiske overvejelser}

\citeauthor{molanderProfesjonsstudierIntroduksjon2008} taler om, at  professionsstudier kan foretages på makro- og mikroniveau.
Idet jeg vil vide noget om, hvordan individuelle praktikere forstår og forvalter deres professionsidentitet, er denne undersøgelse primært på mikroniveau (\citeyear{molanderProfesjonsstudierIntroduksjon2008}, s. 24).
Jeg vil nu kort ridse min tilgang til at besvare min problemformulering op, efterfulgt af en gennemgang af mine metodiske valg.

For at få de udøvende pædagoger i tale har jeg arrangeret et gruppeinterview med to pædagoger, der arbejder på et botilbud for voksne med kognitive funktionsnedsættelser.
Begge pædagoger har lang arbejdserfaring.
Jeg håber på denne måde, at få et perspektiv på (special)pædagogisk faglighed og professionalisme, der er forankret i fagets traditioner.

Min analyse vil tage udgangspunkt i Foucault, og hans beskrivelser af mekanismer, der bidrager til social (selv)styring.
I tråd med den subjektiverede professionsidentitet, kan man så se skjulte magtstrukturer i, hvordan man opretholder en personlig, professionel identitet som pædagog?

Dette vil jeg perspektivere til \citeauthor{baumanLiquidModernity2000}, og hans begreb om nutidens flydende modernitet (\citeyear{baumanLiquidModernity2000}).
Jeg tager afsæt i \citeauthor{kofodOrganisationOgLedelse2016}, der beskriver, hvordan \citeauthor{baumanLiquidModernity2000} kan hjælpe os til at forstå, de krav, der stilles til moderne pædagogiske arbejdspladser \autocite{kofodOrganisationOgLedelse2016}.


\subsection{Den italesatte livsverden}
\citeauthor{tanggaardInterviewetSamtalenSom2015} beskriver interviewet som

\begin{quote}
\ldots en meget almindelig måde, at opnå viden om menneskers livssituation, deres meninger, holdninger og oplevelser \autocite[s. 29]{tanggaardInterviewetSamtalenSom2015}.
\end{quote}

De beskriver videre, at interviewet kan anskues som en af midlerne i den “igangværende individuelle historiefortælling” omkring “selvet og det gode liv” \autocite[s. 30]{tanggaardInterviewetSamtalenSom2015}.
Jeg ønsker netop, i forlængelse af, at livshistorien er en central del af netop velfærdsprofessionernes identitetsdannelse \autocite[s. 457ff]{frederiksenVelfaerdsprofessionerMellemOmsorg2017}, tilgang til mine informanters oplevelser af deres livsverden — deres umiddelbare oplevelser, og ikke mindst, hvordan de italesætter dem \autocite[s. 31]{tanggaardInterviewetSamtalenSom2015}.
Ud fra et socialkonstruktionistisk udgangspunkt vil jeg her se på, hvordan særlige elementer fremhæves som betydningsfulde.
Dermed håber jeg at få øje for, hvad mine informanter anser, som vigtige og meningsbærende i deres oplevede livsverden.

\subsubsection{Hvis livsverden?}
Min empiri består af et gruppeinterview med to erfarne pædagoger.
De fremstår meget homogene — begge er kvinder, begge er over halvtreds, og begge har lang erfaring i specialpædagogisk arbejde.
Dette påvirker givetvis bredden i mit empiriske materiale.
Det kunne især have været ønskeligt, at have haft nogle informanter, med en kortere tilknytning til pædagogfaget.
Desværre var der frafald ved interviewtidspunktet, der forhindrede dette.
Jeg har dermed ikke opnået, hvad \autocite{tanggaardInterviewetSamtalenSom2015} beskriver som idealet — en “mætning” af information omkring pædagogers forholden sig til deres professionsidentitet \autocite[s. 32]{tanggaardInterviewetSamtalenSom2015}.

\subsection{Individet i det postmoderne samfund}

\subsubsection{Individet som forbruger}
\citeauthor{baumanLiquidModernity2000} beskriver i sin udlægning af det postmoderne samfund en overgang fra modernitetens faste tryghed til nutidens flydende modernitet \autocite[s. 2]{baumanLiquidModernity2000}.
Samfundets institutioner mister sin autoritet, og det enkelte individ bliver moralsk ansvarligt for sig selv \autocite[s. 64ff]{baumanLiquidModernity2000}.

Individerne i den flydende modernitet er, fremfor alt, \textit{forbrugere} \autocite[s 73ff; s. 76]{baumanLiquidModernity2000}.
Gennem forbruget køber de sig til identitetsmarkører --- herunder færdigheder til forsørgelse og midler til at overbevise arbejdsgivere at man fortjener en ansættelse \autocite[s. 74]{baumanLiquidModernity2000}.
I følge Bauman er det moderne forbrug ikke funderet i hverken behovsopfyldelse eller indfrielse af lyster - men i opfyldelse af ønsket.
Det er også heri Bauman ser betingelsen for individets frihed --- friheden til “at have identitet” er bundet op i en afhængighed til shopping \autocite[s. 84]{baumanLiquidModernity2000}.

Også på arbejdsmarkedet kan det forbrugende individ shoppe rundt.
Hvis man som arbejder ikke oplever, at ens ønsker og drømme opfyldes på sin arbejdsplads, kan man opløse forholdet og opsøge et nyt, der indehar mere “attraktive” og “spændende” kvaliteter \autocite[s. 169]{kofodOrganisationOgLedelse2016}.
Bauman anfægter, at arbejde i sig selv kan udgøre en forankring for selvforståelse og identitet.
Arbejde i det flydende moderne måles på, hvorvidt det er tilfredsstillende i sig selv; ikke hvor samfundstjenligt det måtte være \autocite[s. 139; 163f]{baumanLiquidModernity2000}.
Men denne frihed medfører også usikkerhed; der i følge Bauman også virker individuerende; i og med at det bryder med den solide modernitets forestillinger om sammenhold og solidaritet \autocite[s. 148]{baumanLiquidModernity2000}.
Og denne opløsning af den gensidige afhængighed gør sig også gældende for arbejdsgiveren.
I den flydende modernitet er det meget nemmere at skifte en arbejder ud med en anden; eller flytte sin produktion til et sted med billigere arbejdskraft, om end det ikke er umiddelbart lige til at eksportere de pædagogiske arbejdspladser \autocite[s. 161]{kofodOrganisationOgLedelse2016}.

\subsubsection{Individuerende selvstyre}

\citeauthor{foucaultOvervagningOgStraf2005} viser i \citetitle{foucaultOvervagningOgStraf2005}, hvordan magtapparaterne ændrer karakter i overgangen til det moderne samfund.
Han viser, hvordan to adskilte styringsmekanismer i den disciplinerede magt sammen producerer et individueret selv \autocite[s. 186]{foucaultOvervagningOgStraf2005}.
Magten virker nu på individniveau — den adskiller, måler, og differentierer ved hjælp af den hierarkiske overvågning, hvor de mange holde øje med de få \autocite[s. 187ff]{foucaultOvervagningOgStraf2005}.
Samtidig virker den normaliserende, hvor sociale sanktioner udskiller og korrigerer afvigelserne \autocite[s. 194]{foucaultOvervagningOgStraf2005}.
På denne måde fremhæver \citeauthor{foucaultOvervagningOgStraf2005}, hvordan “magt” ikke blot er undertrykkende og udslettende, men nu producerer subjekter, individer, der samtidig ligner hinanden \autocite[s. 198]{foucaultOvervagningOgStraf2005}.

For \citeauthor{foucaultSubjectPower1982} er det heller ikke magten, men \textit{subjektet} han har i fokus — blandt andet, hvordan mennesket subjektiverer sig selv. \autocite[s. 777f]{foucaultSubjectPower1982}.
Styring, for \citeauthor{foucaultSubjectPower1982}, er at strukturere andres handlemuligheder, og kun i den udstrækning de er frie til at vælge forskellige handlemuligheder \autocite[s. 790]{foucaultSubjectPower1982}.
Kun det frie menneske kan vælge at deltage i dette spil af magtrelationer, og udskille sig selv som subjekt \autocite[s. 789]{foucaultSubjectPower1982}.

Foucaults udlægning af “guvernementalitet” som en rationalisering af statens virke viser, hvordan overgangen til det moderne samfund også skaber en ny målgruppe for styring: befolkningen.
“Befolkningen” er på samme tid styringens målgruppe; men også dens formål — dennes velfærd, sikkerhed og velbefindende.
Denne guvernementalitet opererer overfor befolkningen, men også usynligt for dem \autocite[s 216ff]{foucaultGovernmentality2000}.

Besynderligt nok forsvinder individet i befolkningens masse i dette perspektiv.
Ikke dermed sagt, at befolkningen ikke består af individer — men det enkelte individ adresseres ikke direkte af guvernementaliteten.
Individet fremtræder kun, som en del af befolkningen som helhed, og har ikke nødvendigvis sammenfaldende interesser med befolkningens ønsker og behov \autocite[s. 217]{foucaultGovernmentality2000}.

\subsubsection{Forbrugende subjekter}

\citeauthor{baumanLiquidModernity2000} går i direkte dialog med styringsmekanismerne der fremlægges af \citeauthor{foucaultOvervagningOgStraf2005}.
Han beskriver det flydende moderne samfund som post-Panoptisk, idet den hierarkiske overvågningsmodel, med den evigt tilstedeværende opsynsmand; nu var opløst.
Nu brillerer opsynsmanden med sin nærmest magiske evne til at flygte fra sit ansvarsområde \autocite[s. 11]{baumanLiquidModernity2000}.
Nu er rollerne vendt på hovedet, og samfundet har nu karakter, af et Synopticon; hvor de mange holder øje med med de få \autocite[s. 85f]{baumanLiquidModernity2000}.

\citeauthor{baumanLiquidModernity2000} tegner et mere pessimistisk billede af individers selvstyring: “den frie vilje” er blot rendyrket styring til lydrørighed i forførende forkledning \autocite[s. 86]{baumanLiquidModernity2000}. 
\citeauthor{foucaultSubjectPower1982} omtaler institutioner som et sted, man med held undersøge sådanne mekanismer fra, idet det giver et “priviligeret observationssted” at se magtudvekslinger under ordnede forhold \autocite[s 791]{foucaultSubjectPower1982}.
Jeg vil derfor undersøge, hvovidt og hvordan disse magtforhold fremtræder for mine informanter på deres arbejdsplads i min analyse.

Hvordan viser disse styringsformer sig i det post-Panoptiske samfund, hvor “frihed”, i følge \citeauthor{baumanLiquidModernity2000} er frihed til, at forbruge?
Hvordan subjektiveres individerne — og subjektiverer individerne sig selv — ind i personalegruppens befolkning? 

\section{Andet arbejde på området}
Det, at beskæftige mig med velfærdsprofessioner i almindelighed eller pædagogprofessionen i særdeleshed er ikke et voldsomt originalt foretagende.
Et lille uddrag af nyligt foretaget arbejde på området følger.

\citeauthor{nielsenAttraktivPaPapiret2017} har undersøgt hvordan mødet med den pædagogiske praksis er for nyuddannede pædagoger i \citetitle{nielsenAttraktivPaPapiret2017}.
De beskriver noget brat overgang og uoverensstemmmelse mellem en tiltagende akademiseret pædagoguddannelse og et relationelt forankret praksisfelt, der omtales som et “praksischok” \autocite{nielsenAttraktivPaPapiret2017}.

I \citetitle{dreyerespersenBekymrendeIdentiteterAnbragte2010} beskriver \citeauthor{dreyerespersenBekymrendeIdentiteterAnbragte2010}, med henvisning til Goffmanns institutionsanalyse, hvordan en oplevelse af særlige arbejdsmiljømæssige vilkår giver anledning til en speciel form for sammenhold og modstandsstrategier i en personalegruppe \autocite{dreyerespersenBekymrendeIdentiteterAnbragte2010}.

\citeauthor{hurFrigorelsensMagt2015} beskriver i \citetitle{hurFrigorelsensMagt2015}, hvordan udviklingen i handicapsynet og handicappolitiken har gjort grænserne for den specialpædagogiske faglighed mere diffus på nogle områder, samtidig med, at den introducerer nye styringsmekanismer overfor de handicappede \autocite{hurFrigorelsensMagt2015}.
Med afsæt i Goffmanns teatermetafor beskriver hun, hvordan udviskningen af institutionsbegrebet har besværliggjort muligheds for at finde en “backstage” til rollen som professionel pædagog.
Hur foretager videre en governmentality-analyse over hvordan en “frihedsdiskurs” omkring handicappede medfører nye og mere skjulte magtformer i specialpædagogisk arbejde.

I \citetitle{meyer-johansenFagligeOrienteringerSocialspecialpaedagogisk2018} undersøger \citeauthor{meyer-johansenFagligeOrienteringerSocialspecialpaedagogisk2018} om der er fællestræk over det brede social- og specialpædagogiske område.
Der afspejles to idealtypiske positioner: En ekspertposition, der både begrunder og legitimerer praksis, og en der vægter en stor relationel viden og forståelse af “personen” over “klienten” \autocite{meyer-johansenFagligeOrienteringerSocialspecialpaedagogisk2018}.

Mit bidrag er et blik for, hvordan erfarne pædagoger taler om sit arbejde — og, ikke mindst, taler sig selv ind i sit arbejde, og gør sit arbejde til en del af sig selv.

\section{Analyse}

Jeg interviewede to erfarne pædagoger, der arbejder på Skovbo, et stort botilbud for voksne mennesker med autismespektrumforstyrrelser og lignende diagnoser.
Skovbo er drevet af Region Nordjylland, og tager pædagogisk afsæt i 
“Gentle Teaching” \autocite{sodisbakkeAutismespektrumforstyrrelser2019}.

I den følgende analyse af mit interview fremgår pædagogerne med initialerne \textbf{AMB} og \textbf{DMC}.

\subsection{Kerneopgaven}

Når jeg spurgte ind til, hvad de anså som deres \textit{kerneopgave}, var der stor sikkerhed:
\begin{description}
\DMC Jamen vi er jo slet ikke i tvivl om, at vores kerneopgave det er beboerne.
Og så derudover, hvis vi laver en cirkel og tager beboerne inde i midten, så er de det vigtigste overhovedet.
Og så der ud fra så kommer en hel masse andre ting som vi skal, ikke også.

\MAA
Kan I uddybe det med beboerne - det giver mening med beboerne; men beboere hvordan? Beboerne på hvilken måde skal man — 

\AMB
Ja vi vil jo allerhelst kunne give beboerne det optimale hver dag. Have tid til at give dem kontakt-ø og have tid til at nå alle deres aktiviteter og de ting der står på deres plan. Men det er jo ikke givet at verden ser sådan ud når man tager på arbejde — ikke hver dag.
\end{description}

Der tegnes et billede af en idealtilstand, hvor man har mulighed for, at gøre “det optimale” for beboerne.
Men der er en samtidig anerkendelse af, at verden ikke er idealtypisk.
Det er ikke givet på forhånd, hvordan verden ser ud — og der er jo også “en hel masse andre” opgaver, der knytter sig til kerneopgaven.

Dette medfører, at der kan gå tid fra, hvad de anser som den egentlige kerneopgave:
\begin{description}

\AMB
Altså, der kan jo være nogle krav, nogle overordnede krav, hvor man tænker jamen hvis vi hele tiden skal dokumentere en hel masse så vi hele tiden skal tjekke det hele dobbelt, ja så er det også tid at jeg føler i hvert fald der går fra beboerne;
Hvor jeg tænker, at så vil jeg heller have tiden ved beboerne og så skulle have noget andet der var dokumentationstiden

\DMC
Ja

\AMB
Men sådan er verden jo ikke lige

\DMC
Næ, og man kan sige, nu i går da havde vi besøg af fødevarekontrollen; og der er så også nogle ting at man skal lave hele tiden; nogle ting som vi skal, ikke også.
Så skal vi dokumentere der; så skal vi dokumentere i Bosted\footnote{Dokumentationssystem på det sociale område}, ikke også

\AMB
Plus de dage hvor der så kommer nogle kontroller, så går der nogen - et personale, nogen fra.
Så er det igen kerneopgaven der må undvære hvad de egentlig skulle have
\end{description}

Der henvises til, en tiltagende administrationsbyrde, der udfordrer professionsudøvelsen \autocite[s. 16]{mik-meyerIndledningSkabeProfessionel2012}.
Dette placeres i en kontekst af, at kommunerne stiller større krav nu, end tidligere — i tråd med pædagogprofessionens manglende autonomi.

Ud over de konkrete dokumentationskrav, er der alt det omkringliggende, der opleves som en tidsrøver.
Det er måske meget smart, at kunne gå tilbage i computersystemet og se en udvikling over tid; men hvad nu hvis den ikke vil samarbejde?\todo{relevans?}

\subsubsection{Struktureret omsorg}
\todo{omsorg/kontrol — coach/træner}
Når de bliver spurgt, om at uddybe, hvad de lægger i, at se beboerne som kerneopgaven, optræder dette som noget naturligt:

\begin{description}
\AMB
jeg tænker da det er naturligt for mig at det er mit job at komme på arbejdet og give dem en god dag fordi det er dem der bor her, og det er deres liv og deres verden man kommer ind i.

\DMC
Ja, give dem omsorg og tryghed og pleje og...

\AMB
Ja

\DMC
og plejen og..

\AMB
for mig er det en naturlig del
\end{description}

Det er omsorgsdimensionen af deres arbejde de lægger vægt på

“omrogrsproblemet” \autocite[s.455ff]{hansbolKonstruktionAfProfessionel2008}

omsorg/kontrol — ser det mere centralt at yde omsorg \autocite[s.??]{frederiksenVelfaerdsprofessionerMellemOmsorg2017}

De træder ind i et andet menneskes hjem \autocite[§ ??]{social-ogindenrigsministerietBekendtgorelseAfLov2019}
Individfokuseret handicappolitik og handicapsyn \autocite{langagerDetAfmalteLiv2013, hurFrigorelsensMagt2015}

Der er dog et normativt, kontrolaspekt:
\begin{description}
  \todo{lidt langt uddrag?}
\AMB
Jeg synes da godt, der kan være dage, hvor dette kan synes værre end andre dage.
At motivere beboerne til det, de skal, fordi selv om vi tenker, vi ved hvad der er godt for dem.
Men hvis de simpelthen ikke vil, så kan de jo være lidt svært at motivere dem til det.

\DMC
Så vil de jo ikke.
Så lykkedes det jo ikke.
Så er det sådan det er.

\AMB
Så det sådan det er.
Og det kan da godt være nogen dage hvor man tænker ahh — øv — det lykkedes ikke.

\DMC
Og der kan også være eksempler hvor man tænker at nu... så gør man alt så godt, men det gik ikke alligevel
Men så er der man skal tænke med vores beboere - det er ikke os der ikke lykkedes - det er fordi de ikke vil noget 

\AMB
og man kan ikke sige, at fordi det lykkedes i går er det ikke sikkert det lykkes i dag
Altså, deres dage er jo lige så forskellige som vores er - det er jo ikke sikker at det lykkes.
\end{description}

Dette individfokuserede perspektiv medvirker også til, at skabe en professionel distance til eventuelle pædagogiske nederlag.

Trods et personligt engagement i opgaven, har man her mulighed for, at ikke tage det professionelle nederlag som et personligt nederlag (men måske et professionelt)
Borgeren ville ikke, og det skal man tage højde for og respektere.
En dobbeltforståelse af borgeren: Den ved ikke, hvad der er det bedste for den; men skal alligevel have lov til at vælge til og fra. 

Den specifikke demografi afdæmper følelser af manglende autonomi, da der er en fagligt funderet forståelse af, at disse borgere profiterer af en nøje tilrettelagt tilgang
\begin{description}

\AMB
jeg synes de fleste dage fungere fint; hvor man også føler man har indflydelse
jeg tænker ikke jeg føler jeg ikke har indflydelse på hvad jeg skal gøre med beboerne selv om de egentlig har faste rammer og fungerer under det så synes jeg stadigvæk at kerneopgaven bliver fulgt ved at de har nogle ting de skal fordi vi ved jo godt inden for den gren vi arbejder på at det er også sådan de har det bedst
Så er det heller ikke fordi man har lyst til at springe ud i verden med dem hver dag eller gøre nogen ting - fordi man godt ved, at det ikke er godt for dem
\end{description}
\todo{uddybes; måske flyttes}

\subsection{Kulturbærende kurser — og en fælles kurs}
Der lægges stor vægt på, at der er et formål, med hvordan arbejdet med borgerne tilrettelægges.

Pædagogerne er selv glade for, at der tilbydes kurser der holder dem opdateret, samtidig med, at der er noget, der er meningsbærende for udførelsen af arbejdet.

De kurser de siger der tilbydes, kan groft set inddeles i to kategorier:

For det første, baggrundsviden, der henvender sig specifikt til den målgruppe, der bor på Skovbo: Autismeforståelse, epilepsi, medicin; men også baggrundsviden omkring strukturelle vilkår for arbejdet: lovgivning og regler.

Og, for det andet, retningslinjer for det pædagogiske arbejde:

Gentle Teaching; en relationel psykologisk-pædagogisk filosofi, der lægger vægt på, at man kun kan udvikle sig hvis man føler sig sikker, elsket, og inkluderet \autocite[s. 3]{mcgeeGentleTeachingNensom2010}

Neuropædagogik; en pædagogisk tilgang, der tager udgangspunkt i individets kognitive forudsætninger, både hvad angår udfordringer og ressourcer \autocite{socialstyrelsenNeuropaedagogik2018}.

De sidste får karakter af ensretning af personalet ud fra en fælles ideologisk forståelse:
\begin{description}

\AMB
Det tænker jeg også det er stor forståelse for. Også fordi, at man er forgang for mange andre i området.
Der er mange, der tænker, det er en god indgangsvinkel at man tænker på alt det positive fremfor at gå ind at tænke man skal lave noget om og man skal være fejlfinder.
Så tænker man i udgangspunktet på det positive man går videre med det positive, og arbejder videre hen i mod jo mere positiv jo mer \ldots godt bliver der, og beboeren hele tiden får smil og omsorg og kærlighed i stedet for at man ligesom prøver at nej nu skal du også gøre det sådan og sådan nu skal du også lave om på det.
Det synes jeg for mig er godt og naturligt. men det gør også at det ikke kun er på arbejdet man bruger det — man bruger det også hjemme, man bruger det også naturligt i sin hverdag og rundtomkring

\DMC
mmm

\MAA
Er det især Gentle Teaching du tænker på der, eller er det også det neuropædagogiske man ser?

\AMB
Jeg synes det er begge dele. Fordi det neuropædagogiske også ser på det hele menneske man  trækker ikke noget ud og skiller det hele ad.
Det synes jeg er en god ting.
Det giver mening.
\end{description}

Begge disse pædagogiske retningslinjer har et udpræget individfokus.
Man skal, via Gentle Teaching, have “sit bedste jeg” med som pædagog, og fokusere på at give beboeren en positiv oplevelse.
Dette understøttes af en helhedlig forståelse af neuropædagogik, hvor man ikke reducerer beboeren til en kognitiv udviklingshæmning, men ser “det hele menneske”

Disse pædagogiske rettesnore fungerer dermed som en slags selvsubjektiverende magtteknologi, hvor det føles rigtigt og naturligt at gøre, hvad der i sidste ende er besluttet på organisationsplan.

Det, at have en fælles idelogisk retningslinje virker også samlende på personalegruppen:

\begin{description}
\MAA
Så det giver, det handler om at skabe en fælles forståelsesramme

\DMC
Ja lige præcis
Og det er også noget man kan mærke når nye mennesker de har været på GT for eksempel, så har de en anden forståelse når de kommer tilbage.
Så er vi nemlig fælles om, at have noget.

\AMB
Men bare også begreberne, at der bliver talt om, hvor man går ind og snakker om forskellige ting, de pædagogiske udtryk de kommer, de bliver brugt mere når man har været på kursus og lige er blevet opdateret.
Så går det mæske lidt i glemmebogen, og så kommer det igen fordi man får det opdateret jævnligt

\DMC
mmm

\AMB
dette synes jeg er en god ting

\MAA
Så der noget med et fælles sprog der også giver en..

\AMB
Ja

\DMC
Der også giver et fællesskab, ja.

\AMB
Både det, og kulturen, at vi hele tiden tænker — vi snakker tit om, at beholde vores kultur og bevare vores kultur, ved ligesom at hele tiden italesætte kulturen;
Hvad er det vi gør, og hvad er det vi er under 
\end{description}

\citeauthor{baumanLiquidModernity2000} understreger, hvor svært det er at skabe tillidsfulde relationer når de er i konstant omvæltning (\citeyear{baumanLiquidModernity2000}, s. 166).
Og når medarbejder og institution ikke er så tæt forbundne som de en gang var — også Skovbo er præget af høj gennemstrømning — kan man se dette som et forsøg på, at skabe et kulturbærende fælles tredje i personalegruppen,  uafhængigt af d enkelte medlemmer.

Ud over den kulturbærende funktion, får de pædagogiske retningslinjer karakter af magtteknologier, der ensretter pædagogernes tilgange.
Gentle Teaching explicitere, det personlige ansvar i, at bringe sig selv i spil overfor beboeren --- på en helt særlig facon.
Samtidigt understreges det individuelle handicapsyn, idet neuropædagogikken har beboerens særegenheder i fokus.

Disse tilgange fremstår også muligvis som attraktive ideologier, netop fordi de skriver sig ind i samtidens indvidualiseringsdiskurs \autocite{andersenUndervisningInstitutionOg2019}.


\subsection{Professionel afgrænsning}
Noget, de adspurgte pædagoger føler, er til endnu større hinder for deres professionelle virke, er, at der ikke er “hænder” nok.
\begin{description}
\DMC
jamen den største forhindring det er jo hvis der ikke er hender nok
Og så prøver vi jo så godt vi kan (suk)
jamen så må vi skære ned på aktiviteterne og 

\AMB
Omdirigere dagen lidt og så skifte lidt så det ikke hver dag er de samme beboere der kommer til at skal undlade nogen af deres aktiviteter
\end{description}

Der er en oplevelse af, at der er svært at rekruttere medarbejdere til Skovbo; både fastansatte og vikarer, hvilket yderligere øger sårbarheden.
Det medfører mange ufaglærte medarbejdere, hvilket opleves soom et nødvendigt onde.
Når de er sammen med mange ufaglærte kollegaer, er der videre en oplevelse af, at være mere “på”:

\begin{description}

\MAA
Det pædagogiske område, både socialpædagogisk og almenpædagogisk, er præget af en hel del ufaglært arbejdskraft. Hvad tænker I om det?

\DMC
Altså, når vi sidder her, som pædagoger, så vil jeg sige, det optimale var da, vi alle sammen var pædagoger. Helt sikkert.

\AMB
Det ville det helt klart. Men det er ikke en mulighed vi har det ved vi jo godt. Fordi der er ikke nok at tage af inden for det her felt.
Og lige indenfor området - det skal jo også gerne være nogen som ved en lille smule om det.
Det er jo sådan noget for det fungerer bedst

\end{description}
\ldots
\begin{description}

\MAA
Kan I mærke en forskel hvis der er mange ufaglærte?

\AMB
Så skal man selv være mere på - jeg synes, at så skal man selv være mere på, obs på at tingene er fulgt op på og sådan noget.
Både med kerneopgaven og også med det faglige.
Man er selv mere på arbejde.
\end{description}

Det er her bemærkelsesværdigt, at der er en adskillelse mellem kerneopgaven — at yde omsorg for beboerne — og “det faglige”.
Omsorgen devalueres i forhold til en mere omfattende “pædagogfaglighed” — men som pædagoger er der et professionelt ansvar for begge dele \autocite[s. ??]{frederiksenVelfaerdsprofessionerMellemOmsorg2017}.

\citeauthor{baumanLiquidModernity2000} henviser til Robert Reich, der har en grov typologi over dagens arbejdsstyrke (\citeyear{baumanLiquidModernity2000}, s. 152):

Symbolanalytikerne; der opfinder og formidler;

Dem, der står for arbejdsstyrkens reproduktion;

Dem, der er beskæftiget med ansigt-til-ansigt personlig service;

Og “rutinearbejderne”; dem, der gør 'alt det andet'

Denne typologi tages også op af \citeauthor{kofodOrganisationOgLedelse2016}, der ser på Bauman i forhold til pædagogisk ledelse.
I hans udlægning lægges pædagogerne ind i gruppe 2 - “velfærdsstatens funktionærer”.
Men samtidig lægges den sociale sektors ansatte i den tredje gruppe \autocite[s. 166]{kofodOrganisationOgLedelse2016}.

For mig at se understreger dette specialpædagogikkens identitetsproblem.
Er de ansat som “velfærdsstatens funktionærer”?
Eller er de blot en del af den sociale sektors ansigt-til-ansigt arbejdere?

De ivaretager og opretholder på den ene hånd samfundets værdier på vegne af velfærdsstaten, og lægger en professionel ære i, at forvalte dette ansvar ordentligt.
På den anden hånd er de til tider svære at skille fra, de ufaglærte rutinearbejdere, der også udgør en stol del af arbejdsstyrken på de specialpædagogiske arbejdspladser. 

De er dog afklaret på, at andre faggrupper har en plads:

\begin{description}
\DMC
Men vi ved jo også, at det er svært overhovedet at få personale her til, ikke også altså.
Men absolut mener jeg vi var ene pædagoger

\AMB
Nu siger du ene pædagoger - jeg synes nu også det er godt, vi har sosu-assistenter

\DMC
Ja ja selvfølgelig - jeg tænker på uddannet personale
\end{description}

Der er en differentiering i forhold til “uddannede” og “uuddannede”.
Det kan godt være, der er forskelle mellem “os med uddannelse”, men disse forskelle udlignes af, den forskel der til “dem uden uddannelse” \autocite[s. 176]{baumanLiquidModernity2000}.
Dette er i tråd med \citeauthor{porsKerneloseKerneopgaverSkolen2015}, hvor den evige afsøgning efter de forskellige faggruppers potentialer afvæbner kampe internt i et felt.
Og det kunne tyde pa, at i den potentialitetsafsøgende velfærdsorganisation har de ufaglærte — trods de er repræsenteret i stort antal — få eller ingen potentielle bidrag.

\subsubsection{Pædagogerne, personalegruppen — og de andre}

Pædagogerne jeg taler med, nævner især om en stigende omfang af dokumentation, som noget der udfordrer professionsudøvelsen.
Men dokumentationen er også en arena, hvor man for alvor skal agere professionelt.
Man optræder på samme tid personligt — kan jeg stå til ansvar for, at dette “ordentligt”? — og professionelt, som repræsentant for botilbudet og pædagogprofessionen.

\begin{description}

\MAA
Hvordan forholder I jer til, til hvem i henvender jer til når I skriver?
Hvem henvender I jer primært til når I skriver?
Hvad tænker i om målgrupper - kan der være flere?

\DMC
Jamen det er der jo, altså, jeg tænker jo når jeg skriver i Bosted, skriver dagsrapporter og sådan, så tænker jeg det er mine kollegaer at jeg skriver til.
Hvis det er status og sådan noget så tænker man jo på, at det både er de pårørende der ser dem men også at det er kommunerne

\MAA
Men man kan jo bede om agtinsigt i sin journal?

\DMC
Jaja

\AMB
Jeg tænker også at det jeg skriver til dagligdags det bliver læst af mange

\DMC
Mmm

\AMB
Både ledelses og region — jeg tænker det bliver læst af mange.
Og det er også vigtigt for mig, at få det skrevet så det lyder ordentligt.

\DMC
Helt sikkert

\AMB
\ldots og ikke kommer nogle dårlige ord deri.
Og hvis der er noget jeg er i tvivl på, så kan jeg godt finde på at sige, hov vil du lige læse det her igennem

\DMC
Ja - "Lyder det her godt nok?"
\end{description}

Der er en klar bevidsthed om, at det skrevne ord ikke blot er til internt brug, men at der er mange øjne der kigger med.
ref: \autocite{hjerrildNarViSkriver2017, andersenUndervisningInstitutionOg2019}

Men der er også et professionelt ansvar, 
\todo{magt/overgreb/guide}
hvad ligger fx i at guide kontra motivere?
\begin{description}
\MAA
Har I eksempler på vendinger, I ikke vil bruge?

\AMB
Nu har vi jo snakket meget omkring om man må 'guide', hvor man heller skal skrive 'motivere'.
På nogen sætninger kan de dog bare virke helt forkert med motivere.
Så skriver man stadigvæk guide.

\DMC
Men det var fordi vi fik at vide ---

\AMB
At det var et dårligt ord, ikke også.

\DMC
Ja, oppe i regionen, at der så de guidning som noget andet end vi gør, ikke også
Og det var noget negativt

\AMB
Der er også forskel på, hvordan man opfanger det der er skrevet, jo
Så hvis man hele tiden skal tænke rundt - først skal man tænke på, om det bliver opfanget positivt , det der står der.

\DMC
Men jeg synes da også som du siger, man tænker på at der kan sidde nogen oppe i regionen, altså ligesom den der kontrol i det

\AMB
Men det skal jo stadigvæk også være noteret alt hvad det er, fordi at så giver det heller ingen mening.
Så er det ingenting værd, hvis ikke det er korrekt det man skriver.
Hvis man kun skriver noget der er pænt.

\DMC
Nej, ja man skal jo skrive det som det er.

\AMB
Det rigtige i forhold til hvad man har oplevet.
\end{description}

\todo{eksperter i borgeren}
De oplever at blive anerkendt som eksperter i borgerne:
\begin{description}
\MAA
Føler i nogle gange, at i bliver underkendte, i forhold til hvad i siger?

\DMC
Nej, det havde jeg ikke tænkt på.

\AMB
Nej. Og det er uanset om det er en neurolog eller en psykiater eller en i samarbejdet fra hjemkommunen.
Jeg synes egentlig altid de er lyttende.

\DMC
Ja, det synes jeg også.

\AMB
Og gode til at komme med spørgsmål og indspark på “hvad er det egentlig vi skal”.

\DMC
Ja det synes jeg også — de lytter meget til hvad vi siger.
\end{description}

Men agerer de måske mere som borgerens stedfortræder i et system det vægter det personlige ansvar?

\subsection{Personligt ejerskab og positiv selvfortælling}

Jeg beskrev ovenfor, hvordan pædagogerne køber ind på den pædagogiske ideologi der uddannes i.

Det, at tage det 'gentle' ind, og gøre det til en del af sig selv kommer også frem således:

\begin{description}

\AMB
Jeg kan godt lide den her at man skal være Gentle; at man skal gå ind at give noget, at der er godt, at man skal tænke det skal være positivt og godt det man gør; fremfor at man skal stå og dunke i bordet næsten og være så stram i det.

\DMC
Jo jo det kan jeg selvfølgelig også.

\AMB
jeg kan godt li den tilgang vi har.
:w
\DMC
Absolut. Det er jo også en del af en selv på en eller anden måde.
Hvordan man selv er, ikke også.

\AMB
og hvis ikke man er det, altså helt ind i kernen fra start af, så tror jeg man bliver det ved at være i det hele tiden.
\end{description}

Det, at ikke “være så stram i det” bliver en del af et positivt selvbillede.
Pædagogerne i karakter af \citeauthor{baumanLiquidModernity2000}s forbrugende arbejdstager kan få en “vare” i form af en særlig identitet som positivt orienteret og med øje for styrker i andre.

Men der er er også noget andet, de adspurgte pædagoger tager til sig i sin livsverden:

\begin{description}

\AMB
I længden
Også fordi det er et presset område i forvejen
Så man skal nok være gjort af et godt stof for at kunne rumme det
\end{description}

Man skal “være gjort af et særligt stof” for at klare skårene.
Her ser man en anden vare, ædagogerne kan vælge til i sit identitetsprojekt: jeg er “en af de seje”.
Dette bliver en anden afgrænsende faktor for en fællesskabsforståelse efter \citeauthor{baumanLiquidModernity2000} — men nu er forskellen “os der blev” kontra “dem der skred”.
Denne afgrænsning har også den fordel, at den har plads til de ufaglærte medarbejdere, og kan skabe sammenhold og tilhørsfornemmelser i den flydende modernitets usikkerhed.

\section{Konklusion}
\todo{rodet. og husk problemformuleringen}
De interviewede pædagoger ser det som deres kerneopgave at være der for beboeren.
Der er ikke et explicit udviklingsperspektiv; men snarere et værdighedsperspektiv, idet amn træder ind i en andens hjem og verden.
Der er også en genspejling af den individualiserede handicapforståelse, hvor man gerne vil beboerne det bedste — men bliver nødt til at acceptere, hvis beboeren fravælger pædagogernes tilbud og tiltag.
Denne individforståelse betyder også, at pædagogerne ikke føler personligt nederlag, hvis de ikke lykkedes.

Kurser og videreuddannelse gør, at de oplever at stå på mere sikker grund i forhold til det skøn og de tvivlsspørgsmål der er et vilkår i professionsudøvelsen.

De “vidensbærende kurser” har en klar funktion — dette er reglerne, dette ved vi om autisme, sådan optræder epilepsi.
De “kulturbærende kurser” får derimod en dobbeltfunktion, hvor det måske er vigtigere, at ideologisk ensrette personalegruppen.
Dette giver fællesskab og fælles retning for personalegruppen som helhed, hvilket kan være en mindst lige så værdifuld funktion, som en eventuel praktisk pædagogisk egnethed.

De adspurgte pædagoger tager i høj grad deres professionsidentitet ind i deres selvforståelse — og ser dette som noget helt selvfølgeligt, og en næsten uungåelig konsekvens af deres professionelle miljø.
En del af denne identtet virker også sammenholdende på personalegruppen — vi er dem, der kan!

Hønen og ægget problematik ift styringsmekanisme-reflektion af samtidens individfokus-tilkøbt identitet som (kærlig, ordentlig, en-af-de-seje)

\section{Perspektivering}
binder an til den omsorgsdiskurs i praksis, som beskrives i \citetitle{nielsenAttraktivPaPapiret2017}.
Det er det centrale for mine informanter, at kunne gøre noget for en anden — helt konkret.

Hur viser hvordan Foucault er anvendelig til, at uddybe konsekvenserne af de seneste årtiers frihedsdiskurs for de handicappede, der har behov for (og får) en træner, hvor der er lagt op til en coach \autocite{hurFrigorelsensMagt2015}.
Mit bidrag er et blik på pædagogernes gøren i praksis.

