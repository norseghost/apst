\section{Indledning}

Velfærdsprofessionerne har længe higet efter anerkendelse og legitimitet.
Om man tager inspiration fra Bourdieus kampe i et felt; eller en Parsonsk funktionalisme; så forekommer disse arbejdsområder som underligt manglefulde i en verden, der stiller krav om “professionalisme”, og hvor identitet bygges op på, hvad man laver.\todo{Bedre slutning i den sætning}

I en Parsonsk professionsforståelse mangler der, at pædagoger, lærere, socialrådgivere og sygeplejersker selv uddanner sine eksperter; selv sætter rammerne for deres arbejde....\todo{mere parsons!}
Fra Bourdieus standpunkt, er disse fagområder besat af kampe om deres ret til at besidde et felt; lige så meget som de forsøger at øge sit ansvarsområde.
Tænk blot tilbage på Reform 14, hvor pædagogerne skulle inddrages i skolen.
Dette er ikke gået helt gnidningsfrit for sig.
Der er også mangel på professionel lukning — en stor del af de, der beskftiger sig med velfærdsstatens kerneydelser er slet ikke uddannet til det! For eksempel, er ???? ud af ???? ansatte på specialpædagogiske opholdssteder ikke pædagoguddannede.

Denne gruppe “fagprofessionelle”\todo{find oprindelse af begreb?} har også en klar udfordring i, at selv kunne sætte rammerne for sit arbejde, og lade disse blive anerkendt som gyldige.
Der er til gengæld ingen mangel på interessenter, der vil give form og struktur til velfærdsprofessionernes arbejde.

For at vende tilbage til det specialpædagogiske område:
\begin{itemize}
  \item
    Der er handleanvisninger fra pædagogiske metoder
  \item
    Den henvisende myndighed søger handling og opfølgning
  \item
    Der er personlige og økonomiske værger ved beslutninger om fx økonomi og sundhed
  \item
    De pårørende “flytter med” forholdsvis ofte, og har også sine egne holdninger
  \item
    Socialtilsynerne og Arbejdsmiljøtilsynet kommer med påtaler og handleanvisninger
  \item
    Der er klare formålsbeskrivelser i Serviceloven, men arbejdet er også underlagt fx forvaltningsloven, offentligshedloven, persondataloven
\end{itemize}

Men dette til trods, tales der stadig om en social- eller specialpædagogisk “faglighed” og “professionalisme”.\todo{bedre overgang til fokus på det specialpædagogiske}
Den er måske lidt diffus og uformelig — profession og faglighed er et tilbagevendende emne i Socialpædagogen — men der synes at være bred enighed (blandt pædagogerne i hvert fald) om at de kan noget særligt og værdifuldt.

Dermed opstår der en række spørgsmål for mig.
Hvordan oplever pædagoger i dag deres faglige identitet?
Hvad er deres oplevelse af professionel autonomi?
Kan de mærke en ændring over tid?
En opsummering til en problemformulering kunne lyde:

Hvordan opleves og opretholdes en (special)pædagogisk faglig identitet og (oplevelse af) autonomi i lyset af styring i det specialpædagogiske arbejde?

Dette kræver en del vidensindsamling.
Der skal afdækkes konkrete styringsmekanismer der har betydning for det specialpædagogiske arbejde, og disse skal placeres i en større samfundsmæssig sammenhæng. I denne proces akter jeg at støtte mig op ad Jennifer O'Day, og hendes accountability-tilgang; samt Campbells kognitive/normative — forrgunds/baggrundsmatrix.

For at bringe de udøvende pædagoger i tale har jeg arrangeret et fokusgruppeinterview, med tre erfarne pædagoger. Jeg vil analysere disse med afsæt i Erving Goffman, og hans beskrivelse af de ansatte på institutionen.\todo{mere at analysere efter? andre teoretiske baggrunde? Goffman har også 'ansigts' begrebet der kunne være relevant}

