
Om man tager inspiration fra Bourdieus kampe i et felt; eller en Parsonsk funktionalisme; så forekommer disse 
I en Parsonsk professionsforståelse mangler der, at pædagoger, lærere, socialrådgivere og sygeplejersker selv uddanner sine eksperter; selv sætter rammerne for deres arbejde....\todo{mere parsons!}
Fra Bourdieus standpunkt, er disse fagområder besat af kampe om deres ret til at besidde et felt; lige så meget som de forsøger at øge sit ansvarsområde.

Der er også mangel på ofessionel lukning — en stor del af de, der beskftiger sig med velfærdsstatens kerneydelser er slet ikke uddannet til det! For eksempel, er ???? ud af ???? ansatte på specialpædagogiske opholdssteder ikke pædagoguddannede.

Denne gruppe “fagprofessionelle”\footnote{“som har eller vedrører specialiseret viden og kompetence som en faguddannet ansat bruger i sit arbejde ofte i modsætning til generalister eller ledere”, \autocite{FagprofessionelDanskeOrdbog}} har også en klar udfordring i, at selv kunne sætte rammerne for sit arbejde, og lade disse blive anerkendt som gyldige.
Velfærdsprofessionerne har længe higet efter anerkendelse.
