
Om man tager inspiration fra Bourdieus kampe i et felt; eller en Parsonsk funktionalisme; så forekommer disse 
I en Parsonsk professionsforståelse mangler der, at pædagoger, lærere, socialrådgivere og sygeplejersker selv uddanner sine eksperter; selv sætter rammerne for deres arbejde....\todo{mere parsons!}
Fra Bourdieus standpunkt, er disse fagområder besat af kampe om deres ret til at besidde et felt; lige så meget som de forsøger at øge sit ansvarsområde.

Der er også mangel på ofessionel lukning — en stor del af de, der beskftiger sig med velfærdsstatens kerneydelser er slet ikke uddannet til det! For eksempel, er ???? ud af ???? ansatte på specialpædagogiske opholdssteder ikke pædagoguddannede.

Her er der tale om en anden form for kamp --- en kamp om positioner i et felt, hvor man afgrænser hvad der har værdi og giver prestige, for dermed at kunne sætte større præg på dagsordenen.

Jeg vil gerne grave lidt dybere i, hvordan udøvende (social)pædagoger ser sig selv som “noget særligt”, der skiller sig ud fra andre faggrupper der arbejder med mennesker med handicap.

Denne gruppe “fagprofessionelle”\footnote{“som har eller vedrører specialiseret viden og kompetence som en faguddannet ansat bruger i sit arbejde ofte i modsætning til generalister eller ledere”, \autocite{FagprofessionelDanskeOrdbog}} har også en klar udfordring i, at selv kunne sætte rammerne for sit arbejde, og lade disse blive anerkendt som gyldige.
Velfærdsprofessionerne har længe higet efter anerkendelse.
o
På det specialpædagogiske område kunne dette for eksempel være:
\begin{itemize}
  \item
    Der er handleanvisninger fra pædagogiske metoder
  \item
    Den henvisende myndighed søger handling og opfølgning — gerne med et økonomisk perspektiv
  \item
    Der er personlige og økonomiske værger, ved beslutninger om fx økonomi og sundhed
  \item
    De pårørende “flytter med” forholdsvis ofte, og har også sine egne holdninger
  \item
    Socialtilsynerne og Arbejdsmiljøtilsynet kommer med påtaler og handleanvisninger
  \item
\end{itemize}\todo{måske mere relevant lidt senere i opgaven?}


Fra en hverdagsopfattelse af professioner er velfærdsprofessionerne — herunder socialpædagoger — tilsyneladende udmærkede professioner. Man har tilegnet sig kompetencer, og lever af at benytte disse. \autocite[ss. 443-444]{frederiksenVelfaerdsprofessionerMellemOmsorg2017}.
Men, som \citeauthor{frederiksenVelfaerdsprofessionerMellemOmsorg2017} beskriver, er denne gruppe “professionelle” ikke helt sammenlignelige med de mere idealtypiske professioner som læger, advokater eller ingeniører.
\todo{dette er meget ufærdigt}

Lønnen er for eksempel ikke særlig høj

Der er en stor del ufaglært arbejdskraft der udfører tilsyneladende de samme opgaver \todo{lukning - frederiksen s 450-ish}


Hertil kommer, hvad \citeauthor{frederiksenVelfaerdsprofessionerMellemOmsorg2017} kalder \textit{omsorgsproblemet} (\citedate[s. 455ff. ]{frederiksenVelfaerdsprofessionerMellemOmsorg2017}.
Det, at omsorgen professionaliseres, og flyttes fra familien til samfundet, medfører en særlig dualitet i arbejdet og professionsidentiteten, hvor man skal navigere mellem omsorg og kontrol \autocite[s. 461]{frederiksenVelfaerdsprofessionerMellemOmsorg2017}.
\todo{etik, normativitet, ligebehandling etc}

\citeauthor{molanderProfesjonsstudierIntroduksjon2008} beskriver professioner som indeholdende performative og organisatoriske aspekter.
Indenfor de performative aspekter finder man normer for, hvad der er gyldig viden, og hvad der er formålstjenlig praksis.
Her er deres skøn - et andet performativt aspekt ved professioner - bundet af udvendige praksisforskrifter og ressourcemæssige hensyn.
Der er begrænset autonomi
erving goffman beskriver, hvordan de ansatte på institutionen danner en særlig kultur og 
En yderligere perspektivering til det omkringliggende samfund 
vil tage udgangspunkt i Campbells kognitive/normative — forgunds/baggrundsmatrix.\todo{her bliver det måske lidt meget forskellig teori på makroniveau?}

Hvordan arter det sig, at der kræves at udvikle en “individuel professionsstil” som (special)pædagog, samtidig med, at man (formodentlig) ønsker, at en personalegruppe kollektivt trækker i samme retning?


